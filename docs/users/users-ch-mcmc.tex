\chapter{Generation of Markov Chains}\label{ch-gmc}
\thispagestyle{headings}
\markboth{Chapter \ref{ch-gmc}: Markov Chain Monte Carlo Methods}{Chapter \ref{ch-gmc}: Markov Chain Monte Carlo Methods}

The PECOS Toolkit currently implements the DRAM algorithm \cite{HaLaMiSa06} for the generation of a Markov chain.
After a review of some concepts and results on Markov chains in Section \ref{sc-gmc-markov-chains},
Section \ref{sc-gmc-dram-alg} details the
Metropolis-Hastings (MH),
Delayed Rejection (DR),
Adaptive Metropolis (AM) and
DRAM algorithms.
Section \ref{sc-gmc-dram-how-to-use} explains how to to develop your ouw application using the DRAM capabilities of the PECOS Toolkit, while
Section \ref{sc-gmc-dram-examples} describes the examples available in the toolkit.
The chapter ends at Section \ref{sc-gmc-planned-features} with a brief list of planned features for next toolkit versions w.r.t. Markov Chain Monte Carlo methods.

\section{Markov Chains}\label{sc-gmc-markov-chains}

This section is based primarily on references \cite{Du05} and \cite{JaPr04}.
The reader can also consult \cite{KaSo05} and \cite{Ro96}.

\subsection{Basic Concepts}

Let $(\Omega,U,P)$ be a probability space and $Y:\Omega\rightarrow S$ be a $U$ measurable r.v.,
that is, a measurable map relative to the $\sigma$-algebras $U$ and $\mathcal{S}$,
as explained in Subsection \ref{subsc-intro-prelim-basic}, page \pageref{subsc-intro-prelim-basic}.
We define
\begin{equation*}
\sigma(Y) = \{A\subset\Omega~:~Y^{-1}(B)=A,\mbox{ for some }B\in\mathcal{S}\}.
\end{equation*}
It is easy to check that:
\[
\begin{array}{rl}
  (i) & \sigma(Y)\mbox{ is a }\sigma\mbox{-algebra},\mbox{ and}\\
 (ii) & \sigma(Y)\mbox{ is the smallest }\sigma\mbox{-algebra}\mbox{ that makes }Y\mbox{ a measurable map}.
\end{array}
\]
We call $\sigma(Y)$ the $\sigma$-algebra generated by $Y$.

Now let $Y:\Omega\rightarrow\mathbb{R}$ be a $U$ measurable scalar r.v..
We will assume that the reader already knows the concept of the
integration of the r.v. $Y$
w.r.t. to its probability distribution $P_Y$
over a Borel set $B\in\mathfrak{B}(\mathbb{R})$ (see, e.g., \cite[Section A.4]{Du05} and \cite[Chapter 9]{JaPr04}).
When such integral exists (it might assume values $\pm\infty$), it will be denoted by one of the following many possible notations:
\begin{equation*}
\int_B yP_Y(dy) = 
\int_B yP(Y^{-1}(dy)) = 
\int_{Y^{-1}(B)}Y(\omega)P(d\omega) =
\int_{Y^{-1}(B)}Y(\omega)dP(\omega).
\end{equation*}
For a vector r.v. $Y:\Omega\rightarrow\mathbb{R}^n$ and $B\in\mathfrak{B}(\mathbb{R}^n)$, the concept is just applied componentwise.

\subsection{Expectation}

The {\it expectation} $E[\mathbf{Y}]$ of a r.v. $\mathbf{Y}:\Omega\rightarrow\mathbb{R}^n$ is defined by
\begin{equation*}
E[\mathbf{Y}] =
\int_{\mathbb{R}^n} \mathbf{y}P_\mathbf{Y}(d\mathbf{y}) = 
\int_{\mathbb{R}^n} \mathbf{y}P(\mathbf{Y}^{-1}(d\mathbf{y})) = 
\int_{\Omega}Y(\omega)P(d\omega) =
\int_{\Omega}Y(\omega)dP(\omega),
\end{equation*}
when such integral exists, including values $\pm\infty$.
A r.v. is called {\it integrable} if and only if its expectation exists and is finite.
We will write $\mathcal{L}^1(\Omega,U,P)$, sometimes just $\mathcal{L}^1$ for short, to denote the vector space of all integrable random variables.
It is easy to check that:
\[
\begin{array}{rl}
  (i) & Y\in\mathcal{L}^1\mbox{ iff }|Y|\in\mathcal{L}^1,                        \\
 (ii) & Y_1=Y_2\mbox{ almost surely (a.s.) }\Rightarrow~E[Y_1]=E[Y_2],\mbox{ and} \\
(iii) & \mbox{the a.s. equality is an equivalence relation between two r.v.}.
\end{array}
\]
For $1 < p < \infty$ we define $\mathcal{L}^p(\Omega,U,P)$ to be the space of r.v. $Y$ such that $|Y|^p\in\mathcal{L}^1$.
For $1\leqslant p < \infty$ we define $L^p(\Omega,U,P)$, sometimes just $L^p$ for short,
to be $\mathcal{L}^p$ module the ``a.s. equality'' equivalence relation, that is,
two elements of $\mathcal{L}^p$ that are a.s. equal are considered to be
representatives (``versions'') of the same element in $L^p$.

\subsection{Conditional Expectation}

\subsection{Transition Probability Kernel}

\subsection{Markov Chain}

\subsection{Variance, Covariance and Covariance Matrix}

\section{The DRAM Algorithm}\label{sc-gmc-dram-alg}

\begin{sidewaystable}[h]
\begin{tabular}{|c||c|c||c|c||c|c|}
\hline
 Initial    & \multicolumn{2}{c||}{Metropolis-Hastings (``MH'')} & \multicolumn{2}{c||}{Delayed Rejection (``DR'')} & \multicolumn{2}{c|}{Adaptive Metropolis (``AM'')} \\
\cline{2-7}
 Position   & candidate     & next position             & candidate     & next position           & candidate     & next position            \\
(iteration) &               &                           & (stage)       &                         &               &                          \\
\hline
\hline
            &               &                           &               &                         &               &                          \\
\cline{4-5}
            &               &                           &               &                         &               &                          \\
\cline{4-5}
            &               &                           & $\vdots$      & $\vdots$                &               &                          \\
\cline{4-5}
            &               &                  & \multicolumn{2}{l||}{until candidate accepted}   &               &                          \\
            &               &                  & \multicolumn{2}{l||}{or maximum stages achieved} &               &                          \\
\hline
\hline
            &               &                           &               &                         &               &                          \\
\cline{4-5}
            &               &                           &               &                         &               &                          \\
\cline{4-5}
            &               &                           & $\vdots$      & $\vdots$                &               &                          \\
\cline{4-5}
            &               &                  & \multicolumn{2}{l||}{until candidate accepted}   &               &                          \\
            &               &                  & \multicolumn{2}{l||}{or maximum stages achieved} &               &                          \\
\hline
\hline
$\vdots$    & $\vdots$      & $\vdots$                  & $\vdots$      & $\vdots$                & $\vdots$      & $\vdots$                 \\
\hline
\hline
            &               &                           &               &                         &               &                          \\
\cline{4-5}
            &               &                           & $\vdots$      & $\vdots$                &               &                          \\
\hline
\hline
            &               &                           &               &                         &               &                          \\
\cline{4-5}
            &               &                           & $\vdots$      & $\vdots$                &               &                          \\
\hline
\hline
$\vdots$    & $\vdots$      & $\vdots$                  & $\vdots$      & $\vdots$                & $\vdots$      & $\vdots$                 \\
\hline
\hline
            &               &                           &               &                         &               &                          \\
\cline{4-5}
            &               &                           & $\vdots$      & $\vdots$                &               &                          \\
\hline
\hline
            &               &                           &               &                         &               &                          \\
\cline{4-5}
            &               &                           & $\vdots$      & $\vdots$                &               &                          \\
\hline
\hline
$\vdots$    & $\vdots$      & $\vdots$                  & $\vdots$      & $\vdots$                & $\vdots$      & $\vdots$                 \\
\hline
\end{tabular}
\caption{Overview of three algorithms for the generation of a Markov chain 
$\{\boldsymbol{\theta}^{(0)},\boldsymbol{\theta}^{(1)},\ldots\}$
: Metropolis-Hastings, Delayed Rejection and Adaptive Metropolis.
Detailed explanations are given in Section \ref{sc-gmc-dram-alg}.}
\label{tab-dram}
\end{sidewaystable}

\subsection{The Metropolis-Hastings (MH) Algorithm}%\label{subsc-gmc-mh-alg}

\subsection{The Delayed Rejeciton (DR) Algorithm}%\label{subsc-gmc-dr-alg}

\subsection{The Adaptive Metropolis (AM) Algorithm}%\label{subsc-gmc-am-alg}

\subsection{The DRAM Algorithm}%\label{subsc-gmc-dram-alg}

\section{How to Use DRAM}\label{sc-gmc-dram-how-to-use}

In order to use the DRAM implementation of the PECOS Toolkit, one needs to take the following steps:
\begin{itemize}
\item {prepare your executable (let us call it ``myappl''):
\begin{itemize}
\item execute ``cd uq/appls/mcmc/'';
\item execute ``cp -R template myappl'';
\item execute ``cd myappl'';
\item change ``template'' and ``Template'' to ``myappl'' and ``Myappl'' in all files, including the file ``Makefile'';
\item code your prior function, if necessary (there is a default prior in the PECOS library);
\item code your likelihood function, which returns a vector of values, each value correspond to the likelihood of a specific output quantity, as explained in Section \ref{sc-intro-qoi}, page \pageref{sc-intro-qoi};
\item execute ``make myappl'';
\end{itemize}
}
\item prepare a file describing the input parameters; let us call it ``myappl.par''; all input parameters are assumed to be scalar r.v.; see Figure \ref{fig-dram-par-file-ex}; 
\item prepare an input file setting the algorithm options; let us call it ``myappl.inp''; see Figure \ref{fig-dram-input-file-ex} and Table \ref{tab-dram-map};
\item execute ``./myappl -i myappl.inp''.
\end{itemize}

\begin{figure}[h]
\begin{verbatim}
# This is the file of input parameters for "myappl".
# Tha name of this file must match the entry "uqParamSpace_inputFile" in
# the input file.
# Lines that begin with the character `#' are considered comment lines.
# Each line that is not a comment line is treated as representing an input
# parameter.
# From left to right, the entries in each input parameter line correspond to:
# --> parameter name (mandatory)
# --> initial value (mandatory)
# --> minimum value (optional; default value to "-inf")
# --> maximum value (optional; default value to "inf")
# --> expectation (optional; default value to "nan")
# --> variance (optional; default value to "inf")
Theta_0 0. -inf
Theta_1 0. -inf inf 0. inf
Theta_2 0.
Theta_3 0. -inf inf 0. inf
# Comment lines can appear anywhere in the file.
# The total number of input parameter lines (4 in this example) must match
# the entry "uqParamSpace_dim" in the input file.
\end{verbatim}
\caption{Example of a file of input parameters for the generation of a Markov Chain}
\label{fig-dram-par-file-ex}
\end{figure}

\begin{figure}[h]
\begin{verbatim}
###############################################
# UQ Parameter Space
###############################################
uqParamSpace_dim       = 4
uqParamSpace_inputFile = uqNormalEx.par

###############################################
# UQ Output Space
###############################################
uqOutputSpace_dim  = 1

###############################################
# UQ DRAM Markov Chain Generator
###############################################
uqDRAM_mh_sizesOfChains           = 5000
uqDRAM_mh_lrUpdateSigma2          = 0
uqDRAM_mh_lrSigma2Priors          = 1.
uqDRAM_mh_lrSigma2Accuracies      = 0.
uqDRAM_mh_lrNumbersOfObs          = 0
uqDRAM_dr_maxNumberOfExtraStages  = 0
uqDRAM_dr_scalesForExtraStages    = 5. 4. 3.
uqDRAM_am_initialNonAdaptInterval = 0
uqDRAM_am_adaptInterval           = 0
uqDRAM_am_sd                      = 1.44
uqDRAM_am_epsilon                 = 1.e-5
uqDRAM_mh_namesOfOutputFiles      = uqNormalExOutput.m
uqDRAM_mh_chainDisplayPeriod      = 500
\end{verbatim}
\caption{Example of an input file for the generation of a Markov Chain}
\label{fig-dram-input-file-ex}
\end{figure}

\begin{sidewaystable}
\begin{tabular}{|l|c|c|c|c|}
\hline
\multicolumn{1}{|c|}{Option Name}        & Explanation       & Default Value & \multicolumn{2}{c|}{Definition} \\
\cline{4-5}
                                         &                   &               & Equation    & Page              \\
\hline
\verb=uqParamSpace_dim=                  &                   &               &             &                   \\
\hline
\verb=uqParamSpace_inputFile=            &                   &               &             &                   \\
\hline
\verb=uqOutputSpace_dim=                 &                   &               &             &                   \\
\hline
\verb=uqDRAM_mh_sizesOfChains=           &                   &               &             &                   \\
\hline
\verb=uqDRAM_mh_lrUpdateSigma2=          &                   &               &             &                   \\
\hline
\verb=uqDRAM_mh_lrSigma2Priors=          &                   &               &             &                   \\
\hline
\verb=uqDRAM_mh_lrSigma2Accuracies=      &                   &               &             &                   \\
\hline
\verb=uqDRAM_mh_lrNumbersOfObs=          &                   &               &             &                   \\
\hline
\verb=uqDRAM_dr_maxNumberOfExtraStages=  &                   &               &             &                   \\
\hline
\verb=uqDRAM_dr_scalesForExtraStages=    &                   &               &             &                   \\
\hline
\verb=uqDRAM_am_initialNonAdaptInterval= &                   &               &             &                   \\
\hline
\verb=uqDRAM_am_adaptInterval=           &                   &               &             &                   \\
\hline
\verb=uqDRAM_am_sd=                      &                   &               &             &                   \\
\hline
\verb=uqDRAM_am_epsilon=                 &                   &               &             &                   \\
\hline
\verb=uqDRAM_mh_namesOfOutputFiles=      &                   &               &             &                   \\
\hline
\verb=uqDRAM_mh_chainDisplayPeriod=      &                   &               &             &                   \\
\hline
\end{tabular}
\caption{Mapping between DRAM options in the input file of Figure \ref{fig-dram-input-file-ex} and the mathematical terms explained in Sections \ref{sc-intro-qoi} and \ref{sc-gmc-dram-alg}.}
\label{tab-dram-map}
\end{sidewaystable}

\section{Examples Provided}\label{sc-gmc-dram-examples}

Three examples related to DRAM are provided: normal target distribution, chemical reactions and algae.
They are discussed in Subsections \ref{subsc-gmc-dram-normal-ex}, \ref{subsc-gmc-dram-himmel-ex} and \ref{subsc-gmc-dram-algae-ex} respectively.

\subsection{Normal Target Distribution}\label{subsc-gmc-dram-normal-ex}

To be explained in the future versions of the documentation.

\subsection{Chemical Reactions}\label{subsc-gmc-dram-himmel-ex}

To be explained in the future versions of the documentation.

\subsection{Algae}\label{subsc-gmc-dram-algae-ex}

\section{Planned Features for Next Releases}\label{sc-gmc-planned-features}
With respect to Markov Chain Monte Carlo methods, the following features are planned for the next versions of the PECOS Toolkit for Predictive Engineering:
\begin{enumerate}
\item capability of running over parallel environments using Trilinos,
\item integration with the DAKOTA Toolkit,
\item chain convergence tests \cite{BrRo98},
\item capability of running over parallel environments using PETSc,
\item hyperprior models,
\item algorithms for multimodal distributions,
\item Gibbs sampler.
\end{enumerate}
