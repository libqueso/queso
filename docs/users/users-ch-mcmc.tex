\chapter{Markov Chain Monte Carlo Methods}\label{ch-mcmc}
\thispagestyle{headings}
\markboth{Chapter \ref{ch-mcmc}: Markov Chain Monte Carlo Methods}{Chapter \ref{ch-mcmc}: Markov Chain Monte Carlo Methods}

The PECOS toolkit currently implements the DRAM algorithm \cite{HaLaMiSa06} for the generation of a Markov chain.
The algorithm is detailed in Section \ref{sc-dram-alg}.
Section \ref{sc-dram-how-to-use} explains how to to develop your ouw application using the DRAM capabilities of the PECOS toolkit, while
Section \ref{sc-dram-examples} describes the examples available in the toolkit.

\section{The DRAM Algorithm}\label{sc-dram-alg}

\begin{sidewaystable}
\begin{tabular}{|c||c|c||c|c||c|c|}
\hline
 Initial    & \multicolumn{2}{c||}{Metropolis-Hastings (``MH'')} & \multicolumn{2}{c||}{Delayed Rejection (``DR'')} & \multicolumn{2}{c|}{Adaptive Metropolis (``AM'')} \\
\cline{2-7}
 Position   & candidate     & next position             & candidate     & next position           & candidate     & next position            \\
(iteration) &               &                           & (stage)       &                         &               &                          \\
\hline
\hline
            &               &                           &               &                         &               &                          \\
\cline{4-5}
            &               &                           &               &                         &               &                          \\
\cline{4-5}
            &               &                           & $\vdots$      & $\vdots$                &               &                          \\
\cline{4-5}
            &               &                  & \multicolumn{2}{l||}{until candidate accepted}   &               &                          \\
            &               &                  & \multicolumn{2}{l||}{or maximum stages achieved} &               &                          \\
\hline
\hline
            &               &                           &               &                         &               &                          \\
\cline{4-5}
            &               &                           &               &                         &               &                          \\
\cline{4-5}
            &               &                           & $\vdots$      & $\vdots$                &               &                          \\
\cline{4-5}
            &               &                  & \multicolumn{2}{l||}{until candidate accepted}   &               &                          \\
            &               &                  & \multicolumn{2}{l||}{or maximum stages achieved} &               &                          \\
\hline
\hline
$\vdots$    & $\vdots$      & $\vdots$                  & $\vdots$      & $\vdots$                & $\vdots$      & $\vdots$                 \\
\hline
\hline
            &               &                           &               &                         &               &                          \\
\cline{4-5}
            &               &                           & $\vdots$      & $\vdots$                &               &                          \\
\hline
\hline
            &               &                           &               &                         &               &                          \\
\cline{4-5}
            &               &                           & $\vdots$      & $\vdots$                &               &                          \\
\hline
\hline
$\vdots$    & $\vdots$      & $\vdots$                  & $\vdots$      & $\vdots$                & $\vdots$      & $\vdots$                 \\
\hline
\hline
            &               &                           &               &                         &               &                          \\
\cline{4-5}
            &               &                           & $\vdots$      & $\vdots$                &               &                          \\
\hline
\hline
            &               &                           &               &                         &               &                          \\
\cline{4-5}
            &               &                           & $\vdots$      & $\vdots$                &               &                          \\
\hline
\hline
$\vdots$    & $\vdots$      & $\vdots$                  & $\vdots$      & $\vdots$                & $\vdots$      & $\vdots$                 \\
\hline
\end{tabular}
\caption{Overview of three algorithms for the generation of a Markov chain 
$\{\boldsymbol{\theta}^{(0)},\boldsymbol{\theta}^{(1)},\ldots\}$
: Metropolis-Hastings, Delayed Rejection and Adaptive Metropolis.
Detailed explanations are given in Section \ref{sc-dram-alg}.}
\label{tab-dram}
\end{sidewaystable}

\section{How to Use DRAM}\label{sc-dram-how-to-use}

In order to use the DRAM implementation of the PECOS toolkit, the steps are:
\begin{itemize}
\item {prepare your executable (let us call it ``myappl''):
\begin{itemize}
\item execute ``cd uq/appls/mcmc/'';
\item execute ``cp -R template myappl'';
\item execute ``cd myappl'';
\item change ``template'' and ``Template'' to ``myappl'' and ``Myappl'' in all files, including the file ``Makefile'';
\item code your prior function, if necessary (there is a default prior in the PECOS library);
\item code your likelihood function, which returns a vector of values, each value correspond to the likelihood of a specific output quantity, as explained in Section \ref{sc-intro-qoi}, page \pageref{sc-intro-qoi};
\item execute ``make myappl'';
\end{itemize}
}
\item prepare a file describing the input parameters; let us call it ``myappl.par''; all input parameters are assumed to be scalar r.v.; see Figure \ref{fig-dram-par-file-ex}; 
\item prepare an input file setting the algorithm options; let us call it ``myappl.inp''; see Figure \ref{fig-dram-input-file-ex} and Table \ref{tab-dram-map};
\item execute ``./myappl -i myappl.inp''.
\end{itemize}

\begin{figure}
\begin{verbatim}
# This the file of input parameters for "myappl".
# Lines that begin with the character `#' are considered comment lines.
# Each line that is not a comment line is treated as corresponding to an
# input parameter.
# From left to right, the entries in each input parameter line correspond to:
# --> parameter name (mandatory)
# --> initial value (mandatory)
# --> minimum value (optional; default value to "-inf")
# --> maximum value (optional; default value to "inf")
# --> expectation (optional; default value to "nan")
# --> variance (optional; default value to "inf")
Theta_0 0. -inf inf
Theta_1 0. -inf inf 0. inf
Theta_2 0. -inf inf
Theta_3 0. -inf inf 0. inf
# Comment lines can appear anywhere in the file.
# The total number of input parameter lines (4 in this example) must match
# the entry "uqParamSpace_dim" in the input file.
\end{verbatim}
\caption{Example of a file of input parameters for the generation of a Markov Chain}
\label{fig-dram-par-file-ex}
\end{figure}

\begin{figure}
\begin{verbatim}
###############################################
# UQ Parameter Space
###############################################
uqParamSpace_dim       = 4
uqParamSpace_inputFile = uqNormalEx.par

###############################################
# UQ Output Space
###############################################
uqOutputSpace_dim  = 1

###############################################
# UQ DRAM Markov Chain Generator
###############################################
uqDRAM_mh_sizesOfChains           = 5000
uqDRAM_mh_lrUpdateSigma2          = 0
uqDRAM_mh_lrSigma2Priors          = 1.
uqDRAM_mh_lrSigma2Accuracies      = 0.
uqDRAM_mh_lrNumbersOfObs          = 0
uqDRAM_dr_maxNumberOfExtraStages  = 0
uqDRAM_dr_scalesForExtraStages    = 5. 4. 3.
uqDRAM_am_initialNonAdaptInterval = 0
uqDRAM_am_adaptInterval           = 0
uqDRAM_am_sd                      = 1.44
uqDRAM_am_epsilon                 = 1.e-5
uqDRAM_mh_namesOfOutputFiles      = uqNormalExOutput.m
uqDRAM_mh_chainDisplayPeriod      = 500
\end{verbatim}
\caption{Example of an input file for the generation of a Markov Chain}
\label{fig-dram-input-file-ex}
\end{figure}

\begin{table}
\begin{tabular}{|l|c|c|c|}
\hline
\multicolumn{1}{|c|}{Option Name}        & Explanation       & \multicolumn{2}{c|}{Definition} \\
\cline{3-4}
                                         &                   & Equation    & Page              \\
\hline
\verb=uqParamSpace_dim=                  &                   &             &                   \\
\hline
\verb=uqParamSpace_inputFile=            &                   &             &                   \\
\hline
\verb=uqOutputSpace_dim=                 &                   &             &                   \\
\hline
\verb=uqDRAM_mh_sizesOfChains=           &                   &             &                   \\
\hline
\verb=uqDRAM_mh_lrUpdateSigma2=          &                   &             &                   \\
\hline
\verb=uqDRAM_mh_lrSigma2Priors=          &                   &             &                   \\
\hline
\verb=uqDRAM_mh_lrSigma2Accuracies=      &                   &             &                   \\
\hline
\verb=uqDRAM_mh_lrNumbersOfObs=          &                   &             &                   \\
\hline
\verb=uqDRAM_dr_maxNumberOfExtraStages=  &                   &             &                   \\
\hline
\verb=uqDRAM_dr_scalesForExtraStages=    &                   &             &                   \\
\hline
\verb=uqDRAM_am_initialNonAdaptInterval= &                   &             &                   \\
\hline
\verb=uqDRAM_am_adaptInterval=           &                   &             &                   \\
\hline
\verb=uqDRAM_am_sd=                      &                   &             &                   \\
\hline
\verb=uqDRAM_am_epsilon=                 &                   &             &                   \\
\hline
\verb=uqDRAM_mh_namesOfOutputFiles=      &                   &             &                   \\
\hline
\verb=uqDRAM_mh_chainDisplayPeriod=      &                   &             &                   \\
\hline
\end{tabular}
\caption{Mapping between DRAM options in the input file of Figure \ref{fig-dram-input-file-ex} and the mathematical terms explained in Sections \ref{sc-intro-qoi} and \ref{sc-dram-alg}.}
\label{tab-dram-map}
\end{table}

\section{Examples Provided}\label{sc-dram-examples}

Three examples related to DRAM are provided: normal target distribution, chemical reactions and algae.
They are discussed in Subsections \ref{subsc-dram-normal-ex}, \ref{subsc-dram-himmel-ex} and \ref{subsc-dram-algae-ex} respectively.

\subsection{Normal Target Distribution}\label{subsc-dram-normal-ex}

\subsection{Chemical Reactions}\label{subsc-dram-himmel-ex}

\subsection{Algae}\label{subsc-dram-algae-ex}

