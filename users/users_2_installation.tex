\chapter{Installation}\label{ch-install}
\thispagestyle{headings}
\markboth{Chapter \ref{ch-install}: Installation}{Chapter \ref{ch-install}: Installation}

%This chapter describes how to install QUESO, test it and use it to create your application.

  
This chapter covers some of the basics that a user will need to know when beginning to use QUESO, such as 
how to obtain, configure, build, install, test \Queso\footnote{For the most up-to-date information,
please visit the online \Queso{} Home Page: \Quesoweb. The most accurate and complete information will be kept at this site.}
and use it to create your application.


        
\section{Pre-QUESO Installation Steps}


Herein, suppose you want to install QUESO and its dependencies on 
\begin{center}
\texttt{/home/username/Installations/} 
\end{center}
%
so that you will not need root access rights. 

There are two main steps to prepare your LINUX computing system  for  the QUESO Library: obtain and install \Queso{} dependencies, and to define a number of environmental variables. These steps are discussed bellow.


\subsection{Obtain and install \Queso{} dependencies}

\Queso{} interfaces to a number of high-quality software packages to provide certain functionality. While some of them are required for the successful installation of \Queso{}, other may be used for enhancing its performance. 

\Queso dependencies are:
\begin{enumerate}%{itemize}

  \item \textbf{STL}: The Standard Template Library is a C++ library of container classes, algorithms, and iterators; it provides many of the basic algorithms and data structures of computer science~\cite{STL}. % The STL is a generic library, meaning that its components are heavily parameterized: almost every component in the STL is a template. You should make sure that you understand how templates work in C++ before you use the STL.

  \item \textbf{GSL}: The GNU Scientific Library is a numerical library for C and C++ programmers. It provides a wide range of mathematical routines such as random number generators, special functions and least-squares fitting. E.g. GSL 1.12~\cite{Gsl}. %There are over 1000 functions in total with an extensive test suite.

  \item \textbf{Boost}: Boost provides free peer-reviewed portable C++ source libraries, which can be used with the C++ Standard Library. E.g. Boost 1.37.0~\cite{Boost}.

  \item \textbf{MPI}: The Message Passing Interface is a standard for parallel programming using the message passing model. E.g. Open MPI~\cite{Openmpi} or MPICH~\cite{Mpich}. 

  \item \textbf{GRVY}: The Groovy Toolkit (GRVY) is a library used to house various support functions often required for application development of high-performance, scientific applications. The library is written in C++, but provides an API for development in C and Fortran. E.g. GRVY 0.31.0 ~\cite{grvy}.

\end{enumerate}%{itemize}

\Queso{} also works with the following optional libraries:

\begin{enumerate}%{itemize}
\item \textbf{HDF5}: The Hierarchical Data Format 5 is a technology suite that makes possible the management of extremely large and complex data collections. E.g. e.g. HDF5  5-1.8.8~\cite{HDF5}.

\item \textbf{GLPK}: The GNU Linear Programming Kit package is is a set of routines written in ANSI C and organized in the form of a callable library for solving large-scale linear programming, mixed integer programming, and other related problems~\cite{GLPK}. 

\item \textbf{PETSc}: The Portable, Extensible Toolkit for Scientific Computation (PETSc) is a suite of data structures and routines for the scalable (parallel) solution of scientific applications modeled by partial differential equations, including parallel linear and nonlinear solvers~\cite{Petsc}.

\item \textbf{Trilinos}: The Trilinos Project is an effort to develop and implement robust algorithms and enabling technologies using modern object-oriented software design, while still leveraging the value of established libraries. It emphasizes abstract interfaces for maximum flexibility of component interchanging, and provides a full-featured set of concrete classes that implement all abstract interfaces.  E.g. Trilinos 9.0.2~\cite{Trilinos}.
\end{enumerate}%{itemize}

PETSc requires MPI for its functionality. \Queso makes use of MPI to when running in parallel for certain operations.
\todo{Is this true? $\Longrightarrow$ } {\color{blue}The parallel version of the library is built upon the foundation provided by MPI. If this library is not available when \Queso{} is configured, only a serial version of \Queso{} can be built.}

% \paragraph*{Note:}
% The basic steps to install GRYV are: 
% \begin{verbatim}
% $ ./configure --prefix=/home/username/Installations/grvy_0_31_0 \
%   --with-boost=/home/username/Installations/boost_1_37_0
% $ make 
% $ make install
% \end{verbatim}
% %
% The directory \texttt{/home/username/Installations/grvy\_0\_31\_0} does not need to exist in advance, since it will be created by the command 'make install' above.



\subsection{Prepare your LINUX environment}

% Step one may differ whether your installation will performed in a stand-alone machine or in a network system which comprises Environment Modules\footnote{\url{http://www.modules.sourceforge.net}}~\footnote{\url{http://www.ices.utexas.edu/sysdocs/linux/modules.html}} to provide easy access to software, such as the one employed in ICES.


Before using QUESO, the user must first set a number of environmental variables, and indicate the full path
of the QUESO's dependencies: GSL, Boost and GVRY. For example, under the UNIX C shell (csh) a command of the form
%export LD_LIBRARY_PATH=\$LD_LIBRARY_PATH:/home/kemelli/LIBRARIES/QUESO_0.45.0/lib
\begin{verbatim}
$ setenv LD_LIBRARY_PATH \$LD_LIBRARY_PATH:
       /home/username/Installations/gsl_1_12/lib:
       /home/username/Installations/boost_1_37_0/lib:
       /home/username/Installations/grvy_0_31_0/lib
\end{verbatim}
can be placed in the user's .bashrc or other startup file. In addition, the user must set the following environmental
variables
\begin{verbatim}
$ setenv CC gcc
$ setenv CXX g++
$ setenv MPICC mpicc
$ setenv MPICXX mpic++
$ setenv F77 f77
$ setenv FC gfortran
\end{verbatim}


Under the UNIX bash shell you may need to replace \texttt{setenv} with \texttt{export}, so a command of the form
\begin{verbatim}
$ setenv CC gcc
\end{verbatim}
is replaced with
\begin{verbatim}
$ export CC=gcc
\end{verbatim}




\section{Obtaining a copy of \Queso{}}

The latest supported public release of \Queso{} is available in the form of a tarball  from \Quesoweb{}.


On most systems, the following commands can be used to expand the tarball:

%
\begin{enumerate}

\item Obtain QUESO

The latest supported public release of QUESO is available from \Quesoweb{}.
For Unix and UNIX-like platforms, it is available in tar format compressed with gzip.  %For Microsoft Windows, it is in ZIP format.

\item {Install QUESO dependencies:

We suggest downloading and installing QUESO and its dependencies as a regular/non-root user, perhaps in \texttt{/home/username/Installations/}. %, as it is a  It is good policy and common practice to place important libraries into the /lib directory.

%install five packages: % item 2
\begin{itemize}
\item GNU Scientific Library (GSL)~\cite{Gsl}, e.g GSL 1.12
\item Boost C++ Libraries~\cite{Boost}, e.g. Boost 1.37.0
\item MPI Library, e.g. Open MPI~\cite{Openmpi} or MPICH~\cite{Mpich}

%The parallel version of the library is built upon the foundation provided by MPI. If this library is not available when \Queso{} is configured, only a serial version of \Queso{} can be built.

\item Trilinos Library~\cite{Trilinos} (optional), e.g. Trilinos 9.0.2
\item HDF5~\cite{HDF5} (optional), e.g. HDF5  5-1.8.8 
%\item {High Performance Computing Toolkit (HPCT)~\cite{Hpct}, e.g. HPCT 0.25.1
\item{ The Groovy Toolkit (GRVY) ~\cite{grvy}, e.g. GRVY 0.31.0
%\begin{itemize}
%\item {
\begin{verbatim}
$ ./configure --prefix=/home/username/Installations/grvy_0_31_0 \
  --with-boost=/home/username/Installations/boost_1_37_0
$ make 
$ make install
\end{verbatim}
%}
%\item 
Note: the directory \texttt{/home/username/Installations/grvy\_0\_31\_0} does not need to exist in advance, since it will be created by the command 'make install' above.
%\end{itemize}
}
\end{itemize}
}





\item {Untar the QUESO tar.gz file (more comments in Section \ref{sc-source-dir-structure}): % item 3

Supposing you have downloaded the file `queso-0.41.0.tar.gz' into \texttt{/home/username/queso\_download/}, follow the steps:
% \item cd /home/username
% \item mkdir queso\_download
% \item cd /home/username/queso\_download
% \item mv $<$ORIGINAL\_LOCATION$>$queso-0.41.0.tar.gz .
% \item tar -zxvf queso-0.41.0.tar

\begin{verbatim}
$ cd /home/username/queso_download/
$ gunzip < queso-0.41.0.tar.gz  | tar xf -
$ cd /home/username/queso_download/queso-0.41.0   #enter the folder 
\end{verbatim}
}


\item {Configure the QUESO building environment: % item 4

\Queso{} uses the GNU autoconf system for configuration, which detects various features of the host system and creates the Makefiles. %On most systems it should be sufficient to say:
% \begin{verbatim}
% $ ./configure         
% \end{verbatim}
%  
%             Or
%             $ sh configure
% 
The configuration process can be controlled through environment variables, command-line switches, and host configuration files.
For a complete list of switches type:
\begin{verbatim}
$ ./configure  --help       
\end{verbatim}
%
 The default installation location is `\texttt{/usr/local/bin}', `\texttt{/usr/local/lib}' etc., which requires superuser privileges. To use a path
        other than `\texttt{/usr/local}', specify the path with the `\texttt{--prefix=PATH}' switch. For instance `\texttt{--prefix=\$HOME}'.



The basic steps to configure QUESO using GRVY, Boost, Trilinos and GSL for installation at `\texttt{/home/username/Installations/queso\_0\_41\_0\_gnu}' are:
\begin{verbatim}
$ ./configure --prefix=/home/username/Installations/queso_0_41_0_gnu \
  --with-trilinos=/home/username/Installations/trilinos_9_0_2 \
  --with-boost=/home/username/Installations/boost_1_37_0 \
  --with-gsl-prefix=/home/username/Installations/gsl_1_12 \
  --with-gvry=/home/username/Installations/grvy_0_31_0 \
  CXXFLAGS=''-DMPICH_IGNORE_CXX_SEEK -O3 -Wall -wd383 -wd981 -wd1572''
\end{verbatim}

Note: the directory `\Verb+/home/username/Installations/queso_0_41_0_gnu+' does not need to exist in advance, since it will be created in step 7.
}
%verbatim: '\Verb+' -> begin of environment
%          '+'      -> end of environment

\item {Compile the \Queso{} source code % (library, examples and tests): % item 5

The library, confidence tests, and programs can be built by saying just:
\begin{verbatim}
$ make
\end{verbatim}
}

\item {Check the compiled source (more comments in Section \ref{sc-checks})%: % item 6

\Queso{} comes with various test suites, all of which can be run by:
\begin{verbatim}
$ make check
\end{verbatim}
}

\item {Install the QUESO library (more comments in Section \ref{sc-installed-dir-structure}): % item 7

The \Queso{} Library, include files, and support programs can be installed by: %in a (semi-~)public place by:
\begin{verbatim}
$ make install 
\end{verbatim}}

The files are installed under the directory specified with `\texttt{--prefix=DIR}' in Step 4. The directories, if not existing, will be
        created automatically.%, provided the mkdir command supports the -p  option.

\item {create the documentation in html format: % item 8
\begin{verbatim}
$ make docs
\end{verbatim}

A folder named \Verb+docs+ will be created in \Verb+/home/username/queso_download/queso-0.41.0+ (your current path) and you may access the code documentation in two different ways:
\begin{enumerate}
\item HyperText Markup Language (HTML)  format:
\begin{verbatim}
$ cd docs/html
$ firefox index.html
\end{verbatim}

\item Portable Document Format (PDF) format:
\begin{verbatim}
$ cd docs
$ acroread queso.pdf
\end{verbatim}

\end{enumerate}


}
\end{enumerate}
%%
%These steps are described in Sections~\ref{section:download}
%through~\ref{section:compile}.

%\section{Download} \label{section:download}
%There are two methods for obtaining the QUESO source code: the PECOS
%subversion repository and downloading the latest released version from
%?? (FIX ME: Does a download location exist?).  If you have read access
%to the PECOS subversion repository, you may obtain the code directly
%from the repository:
%%
%\begin{verbatim}
%svn co https://svn.ices.utexas.edu/repos/pecos/uq/queso <WORK_DIR>
%\end{verbatim}
%%
%where \verb+<WORK_DIR>+ denotes the desired download location.  If you
%do not have access to the repository, the latest QUESO release may be
%downloaded from ??.  After
%downloading the source tarball to \verb+<WORK_DIR>+, unpack the source
%as follows:
%%
%\begin{verbatim}
%cd <WORK_DIR>
%tar -zxvf queso_0.41.0.tar.gz (FIX ME: check tarball name)
%\end{verbatim}
%%

%\section{Configure the Build Environment} \label{section:configure}
%After downloading the source code, move into the top-level QUESO
%directory to configure the build environment.  Depending on how you
%obtained the source, this directory will be different.  If you
%obtained the source from the subversion repository,
%%
%\begin{verbatim}
%cd <WORK_DIR>/branches/0.41.0
%\end{verbatim}
%%
%If you downloaded the source from ?? and unpacked the tarball,
%%
%\begin{verbatim}
%cd <WORK_DIR>/queso_0.41.0
%\end{verbatim}
%%

%To configure the build environment, the \verb+configure+ script will
%be run.  The \verb+configure+ script is not stored on the subversion
%repository.  Thus, you may need to generate it using the
%\verb+bootstrap+ utility:
%%
%\begin{verbatim}
%./bootstrap
%\end{verbatim}
%%
%Then, to prepare the build environment, run
%%
%\begin{verbatim}
%./configure
%\end{verbatim}
%%
%The \verb+configure+ script accepts many options that you may use to
%customize the build environment.  For example, specifying
%\verb+--prefix=<INSTALL_DIR>+ sets \verb+<INSTALL_DIR>+ as the
%installation location used by the \verb+make install+ target, and
%specifiying \verb+--with-trilinos=<TRI_DIR>+ indicates that
%\verb+<TRI_DIR>+ is the root directory of your Trilinos installation.
%You may also use environment variables---e.g., \verb+CXX+ and
%\verb+CXXFLAGS+---to override default choices made by the
%\verb+configure+ script.

%To see the full list of configure options for item 3 above, use
%\begin{verbatim}
%./configure --help
%\end{verbatim}
%%

%\section{Compile the Code} \label{section:compile}
%After successfully running \verb+configure+, type
%%
%\begin{verbatim}
%make
%make install
%\end{verbatim}
%%
%to build and install the QUESO Tool libraries, headers, and examples.

\section{The Source Directory Structure} \label{sc-source-dir-structure}

The QUESO source directory contains three main directories. They are listed below and more
information about them can be obtained with the html documentation from step 8 above:
\begin{itemize}
\item {'libs', with five subdirectories:
\begin{itemize}
\item 'libs/core/', with 'inc' and 'src' subdirectories,
\item 'libs/misc/', with 'inc' and 'src' subdirectories,
\item 'libs/basic/', with 'inc' and 'src' subdirectories,
\item 'libs/stats/', with 'inc' and 'src' subdirectories, and
\item 'libs/interface/'.
\end{itemize}
}
\item {'examples', with four subdirectories:
\begin{itemize}
\item 'examples/statisticalForwardProblem/',
\item 'examples/statisticalInverseProblem1/',
\item 'examples/validationCycle/', and
\item 'examples/validationCycle2/'.
\end{itemize}
}
\item {'test', with four subdirectories:
\begin{itemize}
\item 'test/t01\_valid\_cycle/',
\item 'test/t02\_sip\_sfp/',
\item 'test/t03\_sequence/', and
\item 'text/gsl\_tests'.
\end{itemize}
}
\end{itemize}

The executables under 'examples/validationCycle2/', 'test/t02\_sip\_sfp/', 'test/t03\_sequence/' and 'test/gsl\_tests/'
have the majority of their codes in *.C files.
They might then be easier to understand than
the other exectuables in 'examples' and 'test/t01\_valid\_cycle', which
have the majority of their codes in *.h files, with templated routines.
It should be clear, though, that all executables might be implemented in either *.h or *.C files.
It is a matter of how generic you want your application to be.

\section{Checking the Compiled Source} \label{sc-checks}

Just run 'make ckeck' at the same directory where 'configure' and 'make' were run.
Many printouts will appear in the screen, but towards the end of them you should see
a message like:
\begin{verbatim}
==================
All 2 tests passed
==================
\end{verbatim}

The 2 tests mentioned in this message are the ones under 'test/t01\_valid\_cycle' and
'test/t02\_sip\_sfp'. These tests are used as part of the periodic QUESO regression tests.
The code for 't02\_sip\_sfp' is mentioned in Subsection \ref{subsc-t02} and is explained in more detail in Chapter \ref{ch-appl-example}.

\section{Running the Executables Provided with QUESO} \label{sc-running-execs}

This section assumes that you have successfully executed steps 1 through 6 above.
The codes listed in this section have explanations inside themselves, and some of them
print messages during execution to make it clearer what is going on.

\subsection{Executable at 'examples/statisticalInverseProblem1/'}

Just run the following commands:
\begin{itemize}
\item cd /home/username/queso\_download/queso-0.41.0/
\item cd examples/statisticalInverseProblem1/tests/test\_2009\_02\_03/
\item rm outputData/*
\item ../../src/exStatisticalInverseProblem1\_gsl sip.inp [this will take some seconds]
\item matlab
\item {[inside matlab]} sip\_plot
\item {[press the left button of the mouse at a picture displayed by 'sip\_plot.m', in order to display the next picture]}
\item {[inside matlab]} exit
\item ls -l outputData/*.png
\end{itemize}

\subsection{Executable at 'examples/statisticalForwardProblem/'}

Just run the following commands:
\begin{itemize}
\item cd /home/username/queso\_download/queso-0.41.0/
\item cd examples/statisticalForwardProblem1/tests/test\_2009\_02\_11/
\item rm outputData/*
\item ../../src/exStatisticalForwardProblem1\_gsl sfp.inp [this will take some seconds]
\item matlab
\item {[inside matlab] sfp\_plot}
\item {[press the left button of the mouse at a picture displayed by 'sfp\_plot.m', in order to display the next picture]}
\item {[inside matlab]} exit
\item ls -l outputData/*.png
\end{itemize}

%\subsection{Executable at 'examples/validationCycle/'}

%$~$\\

\subsection{Executable at 'test/t02\_sip\_sfp/sip\_sfp/'}\label{subsc-t02}

Just run the following commands:
\begin{itemize}
\item cd /home/username/queso\_download/queso-0.41.0/
\item cd test/t02\_sip\_sfp/sip\_sfp/
\item rm outputData/*
\item ./SipSfpExample\_gsl example.inp [this will take some seconds]
\item matlab
\item {[inside matlab]} example\_plots
\item {[press the left button of the mouse at a picture displayed by 'example\_plots.m', in order to display the next picture]}
\item {[inside matlab]} exit
\item ls -l outputData/*.png
\end{itemize}

\section{The Installed Directory Structure} \label{sc-installed-dir-structure}

This section assumes you have successfully executed steps 1 through 7 above.
The QUESO installed directory contains three main directories:
\begin{itemize}
\item 'lib',
\item 'include', and
\item {'examples', with two subdirectories:
\begin{itemize}
\item 'examples/basic/',
\item 'examples/advanced/'.
\end{itemize}
}
\end{itemize}

\section{Create your Application with the installed QUESO} \label{sc-use-queso}

Prepare your environment by running
\begin{verbatim}
setenv LD_LIBRARY_PATH \$LD_LIBRARY_PATH:
       /home/username/Installations/queso_0_41_0_gnu/lib
\end{verbatim}

An example Makefile is given below:
\begin{verbatim}
# BEGIN OF MAKEFILE
QUESO_DIR = /home/username/Installations/queso_0_41_0_gnu/
TRILINOS_DIR = /home/username/Installations/trilinos_9_0_2/
BOOST_DIR = /home/username/Installations/boost_1_37_0/
GSL_DIR = /home/username/Installations/gsl_1_12/
HPCT_DIR = /home/username/Installations/hpct_0_25_1/

include $(TRILINOS_DIR)/include/Makefile.export.epetra

INC_PATHS = \
	-I. \
	-I$(QUESO_DIR)/include \
	-I$(MPI_DIR)/include \
	-I$(BOOST_DIR)/include/boost_1_37_0 \
	-I$(GSL_DIR)/include \
	-I$(HPCT_DIR)/include \
	$(EPETRA_INCLUDES)

LIBS = \
	-L$(QUESO_DIR)/lib \
	-lqueso \
	-L$(MPI_DIR)/lib \
	-L$(TRILINOS_DIR)/lib \
	-L$(BOOST_DIR)/lib \
	-lboost_program_options \
	-L$(GSL_DIR)/lib \
	-lgsl \
	-L$(HPCT_DIR)/lib \
	-lhpct \
	$(EPETRA_LIBS)

CXX = mpic++
CXXFLAGS += -O3 -Wall -c

default: all

.SUFFIXES: .o .C

all:	ex_gsl

clean:
	rm -f *~
	rm -f *.o
	rm -f example

ex_gsl: example_main.o example_likelihood.o example_qoi.o example_compute.o
	$(CXX) example_main.o \
	       example_likelihood.o \
	       example_qoi.o \
	       example_compute.o \
	       -o example_gsl $(LIBS)

%.o: %.C
	$(CXX) $(INC_PATHS) $(CXXFLAGS) $<
# END OF MAKEFILE
\end{verbatim}

More documentation is provided in Chapter \ref{ch-appl-example}.
