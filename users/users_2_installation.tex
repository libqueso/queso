\chapter{Installation}\label{ch-install}
\thispagestyle{headings}
\markboth{Chapter \ref{ch-install}: Installation}{Chapter \ref{ch-install}: Installation}

There are nine steps to make the QUESO Library available at your LINUX computing system.
They are listed below, with examples of commands:
%
\begin{enumerate}
\item {install five packages:
\begin{itemize}
\item GNU Scientific Library (GSL)~\cite{gsl},
\item Boost C++ Libraries~\cite{boost}, 
\item MPI Library, e.g. Open MPI~\cite{Openmpi} or MPICH~\cite{Mpich},
\item Trilinos Library~\cite{Trilinos} and
\item High Performance Computing Toolkit (HPCT)~\cite{hpct}.
\end{itemize}
}
\item {prepare your LINUX environment (assuming csh; some commands might be enough):
\begin{itemize}
\item module load gnu
\item module load openmpi
\item {
\begin{verbatim}
setenv LD_LIBRARY_PATH \$LD_LIBRARY_PATH:
       /home/johndoe/Installations/gsl_1_12/lib:
       /home/johndoe/Installations/Boost_1_37_0/lib:
       /home/johndoe/Installations/hpct/lib
\end{verbatim}
}
\item setenv CC gcc
\item setenv CXX g++
\item setenv MPICC mpicc
\item setenv MPICXX mpic++
\item setenv F77 f77
\item setenv FC gfortran
\end{itemize}
}
\item {untar the QUESO tar.gz file (more comments in Section \ref{sc-dir-structure}):
\begin{itemize}
\item cd /home/johndoe
\item mkdir queso\_download
\item cd /home/johndoe/queso\_download
\item mv $<$ORIGINAL\_LOCATION$>$queso-0.4.1.tar.gz .
\item tar -zxvf queso-0.4.1.tar
\end{itemize}
}
\item {configure the QUESO building environment:
\begin{itemize}
\item cd /home/johndoe/queso\_download/queso-0.4.1
\item ./bootstrap
\item {
\begin{verbatim}
./configure --prefix=/home/johndoe/Installations/queso_0_4_1_gnu \
  --with-trilinos=/home/johndoe/Installations/trilinos_9_0_2 \
  --with-boost=/home/johndoe/Installations/1.37.0 \
  --with-gsl-prefix=/home/johndoe/Installations/gsl_1_12 \
  --with-hpct-prefix=/home/johndoe/Installations/hpct \
  CXXFLAGS=''-DMPICH_IGNORE_CXX_SEEK -O3 -Wall -wd383 -wd981''
\end{verbatim}
}
\item if you want to see the full list of configure options, just run ``./configure --help''
\end{itemize}
}
\item {compile the QUESO code (library, examples and tests):
\begin{itemize}
\item make
\end{itemize}
}
\item {check the installation (more comments in Section \ref{sc-checks}):
\begin{itemize}
\item make check
\end{itemize}
}
\item {install the QUESO library:
\begin{itemize}
\item make install
\end{itemize}
}
\item {create the documentation in html format:
\begin{itemize}
\item make html
\item firefox doxygen-doc/html/index.html
\end{itemize}
}
\item {execute your own application (more comments in Chapter \ref{ch-appl-example}):
\begin{itemize}
\item code your program
\item setenv LD\_LIBRARY\_PATH \$LD\_LIBRARY\_PATH:/home/johndoe/Installations/queso\_0\_4\_1\_gnu/lib
\item edit your Makefile
\item compile and link your program
\item execute your program
\end{itemize}
}
\end{enumerate}
%%
%These steps are described in Sections~\ref{section:download}
%through~\ref{section:compile}.

%\section{Download} \label{section:download}
%There are two methods for obtaining the QUESO source code: the PECOS
%subversion repository and downloading the latest released version from
%?? (FIX ME: Does a download location exist?).  If you have read access
%to the PECOS subversion repository, you may obtain the code directly
%from the repository:
%%
%\begin{verbatim}
%svn co https://svn.ices.utexas.edu/repos/pecos/uq/queso <WORK_DIR>
%\end{verbatim}
%%
%where \verb+<WORK_DIR>+ denotes the desired download location.  If you
%do not have access to the repository, the latest QUESO release may be
%downloaded from ??.  After
%downloading the source tarball to \verb+<WORK_DIR>+, unpack the source
%as follows:
%%
%\begin{verbatim}
%cd <WORK_DIR>
%tar -zxvf queso_0.4.1.tar.gz (FIX ME: check tarball name)
%\end{verbatim}
%%

%\section{Configure the Build Environment} \label{section:configure}
%After downloading the source code, move into the top-level QUESO
%directory to configure the build environment.  Depending on how you
%obtained the source, this directory will be different.  If you
%obtained the source from the subversion repository,
%%
%\begin{verbatim}
%cd <WORK_DIR>/branches/0.4.1
%\end{verbatim}
%%
%If you downloaded the source from ?? and unpacked the tarball,
%%
%\begin{verbatim}
%cd <WORK_DIR>/queso_0.4.1
%\end{verbatim}
%%

%To configure the build environment, the \verb+configure+ script will
%be run.  The \verb+configure+ script is not stored on the subversion
%repository.  Thus, you may need to generate it using the
%\verb+bootstrap+ utility:
%%
%\begin{verbatim}
%./bootstrap
%\end{verbatim}
%%
%Then, to prepare the build environment, run
%%
%\begin{verbatim}
%./configure
%\end{verbatim}
%%
%The \verb+configure+ script accepts many options that you may use to
%customize the build environment.  For example, specifying
%\verb+--prefix=<INSTALL_DIR>+ sets \verb+<INSTALL_DIR>+ as the
%installation location used by the \verb+make install+ target, and
%specifiying \verb+--with-trilinos=<TRI_DIR>+ indicates that
%\verb+<TRI_DIR>+ is the root directory of your Trilinos installation.
%You may also use environment variables---e.g., \verb+CXX+ and
%\verb+CXXFLAGS+---to override default choices made by the
%\verb+configure+ script.

%To see the full list of configure options for item 3 above, use
%\begin{verbatim}
%./configure --help
%\end{verbatim}
%%

%\section{Compile the Code} \label{section:compile}
%After successfully running \verb+configure+, type
%%
%\begin{verbatim}
%make
%make install
%\end{verbatim}
%%
%to build and install the QUESO Tool libraries, headers, and examples.

\section{Directory Structure} \label{sc-dir-structure}

The QUESO library contains three main directories. They are listed below and more
information about them can be obtained with the html documentation (see Section \ref{sc-html}):
\begin{itemize}
\item {'libs', with five subdirectories:
\begin{itemize}
\item 'libs/core/', with 'inc' and 'src' subdirectories,
\item 'libs/misc/', with 'inc' and 'src' subdirectories,
\item 'libs/basic/', with 'inc' and 'src' subdirectories,
\item 'libs/stats/', with 'inc' and 'src' subdirectories, and
\item 'libs/interface/'.
\end{itemize}
}
\item {'examples', with three subdirectories:
\begin{itemize}
\item 'examples/statisticalForwardProblem/',
\item 'examples/statisticalInverseProblem1/', and
\item 'examples/validationCycle/'.
\end{itemize}
}
\item {'test', with three subdirectories:
\begin{itemize}
\item 'test/t01\_valid\_cycle/',
\item 'test/t02\_sip\_sfp/', and
\item 'test/t03\_sequence/'.
\end{itemize}
}
\end{itemize}

\section{Checking the Installation} \label{sc-checks}

Just run 'make ckeck' at the same directory where 'configure' and 'make' were run.
Many printouts will appear in the screen, but towards the end of them you should see
a message like:
\begin{verbatim}
==================
All 2 tests passed
==================
\end{verbatim}

The 2 tests mentioned in this message are the ones under 'test/t01\_valid\_cycle' and
'test/t02\_sip\_sfp'. These tests are used as part of the periodic QUESO regression tests.
The code for 't02\_sip\_sfp' is explained in Chapter \ref{ch-appl-example}.
