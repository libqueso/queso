\chapter{Installation}\label{ch-install}
\thispagestyle{headings}
\markboth{Chapter \ref{ch-install}: Installation}{Chapter \ref{ch-install}: Installation}

The QUESO Tool currently runs over uni- and multi-processor Linux
environments and requires the following packages to be installed in
your computing system:
%
\begin{itemize}
\item GNU Scientific Library (gsl)~\cite{gsl},
\item Boost C++ Libraries~\cite{boost}, 
\item an MPI implementation---e.g., Open MPI~\cite{Openmpi} or MPICH~\cite{Mpich}---and
\item Trilinos (TO DO: cite specific packages?)~\cite{Trilinos}.
\end{itemize}
%
If you do not have these packages, you should download and install
them before continuing.

There are three steps to building the QUESO Tool:
%
\begin{itemize}
\item download the code,
\item prepare the build environment, and
\item compile.
\end{itemize}
%
These steps are described in Sections~\ref{section:download}
through~\ref{section:compile}.

\section{Download} \label{section:download}
There are two methods for obtaining the QUESO source code: the PECOS
subversion repository and downloading the latest released version from
?? (FIX ME: Does a download location exist?).  If you have read access
to the PECOS subversion repository, you may obtain the code directly
from the repository:
%
\begin{verbatim}
svn co https://svn.ices.utexas.edu/repos/pecos/uq/queso <WORK_DIR>
\end{verbatim}
%
where \verb+<WORK_DIR>+ denotes the desired download location.  If you
do not have access to the repository, the latest QUESO release may be
downloaded from ??.  After
downloading the source tarball to \verb+<WORK_DIR>+, unpack the source
as follows:
%
\begin{verbatim}
cd <WORK_DIR>
tar -zxvf queso_0.4.1.tar.gz (FIX ME: check tarball name)
\end{verbatim}
%

\section{Configure the Build Environment} \label{section:configure}
After downloading the source code, move into the top-level QUESO
directory to configure the build environment.  Depending on how you
obtained the source, this directory will be different.  If you
obtained the source from the subversion repository,
%
\begin{verbatim}
cd <WORK_DIR>/branches/0.4.1
\end{verbatim}
%
If you downloaded the source from ?? and unpacked the tarball,
%
\begin{verbatim}
cd <WORK_DIR>/queso_0.4.1
\end{verbatim}
%

To configure the build environment, the \verb+configure+ script will
be run.  The \verb+configure+ script is not stored on the subversion
repository.  Thus, you may need to generate it using the
\verb+bootstrap+ utility:
%
\begin{verbatim}
./bootstrap
\end{verbatim}
%
Then, to prepare the build environment, run
%
\begin{verbatim}
./configure
\end{verbatim}
%
The \verb+configure+ script accepts many options that you may use to
customize the build environment.  For example, specifying
\verb+--prefix=<INSTALL_DIR>+ sets \verb+<INSTALL_DIR>+ as the
installation location used by the \verb+make install+ target, and
specifiying \verb+--with-trilinos=<TRI_DIR>+ indicates that
\verb+<TRI_DIR>+ is the root directory of your Trilinos installation.
You may also use environment variables---e.g., \verb+CXX+ and
\verb+CXXFLAGS+---to override default choices made by the
\verb+configure+ script.  To see the full list of options, use
%
\begin{verbatim}
./configure --help
\end{verbatim}
%

\section{Compile the Code} \label{section:compile}
After successfully running \verb+configure+, type
%
\begin{verbatim}
make
make install
\end{verbatim}
%
to build and install the QUESO Tool libraries, headers, and examples.
