\chapter{C++ Classes in the Library}\label{ch-classes}
\thispagestyle{headings}
\markboth{Chapter \ref{ch-classes}: C++ Classes in the Library}{Chapter \ref{ch-classes}: C++ Classes in the Library}

%The QUESO-MCMC Tool currently implements the DRAM algorithm \cite{HaLaMiSa06} for the generation of a Markov chain.
%Section \ref{sc-gmc-eight-steps} explains how to develop your own application using the DRAM capabilities of the QUESO-MCMC Tool, while
%Section \ref{sc-gmc-dram-output} describes the output information generated by the toolkit and
%Sections
%\ref{sc-gmc-dram-normal-ex},
%\ref{sc-gmc-dram-chem-ex} and
%\ref{sc-gmc-dram-algae-ex}
%describe the three examples available,
%all of them also available in \cite{mcmctool}.

%The chapter ends at Section \ref{sc-gmc-planned-features} with a brief list of planned features for next toolkit versions w.r.t. Markov Chain Monte Carlo methods.

\section{Core Classes}



Classes in QUESO can be divided in four main groups:
\begin{description}
\item[ core:] environment (and options), vector, matrix;

\item[ templated basic:] vector sets (and subsets, vector spaces),  scalar function, vector function, scalar sequence, vector sequence;

The templated basic classes are necessary for the definition and description of other entities, such as RVs, Bayesian solutions of IPs, sampling algorithms and chains.

%\item Vector sets, subsets and spaces (see Figure \ref{fig-vector-space-subset-classes}),
%\item Scalar function (see Figure \ref{fig-scalar-function-class}),
%\item Vector function (see Figure \ref{fig-vector-function-class}),
%\item Scalar sequence (see Figure \ref{fig-scalar-sequence-class}), and
%\item Vector sequence (see Figure \ref{fig-vector-sequence-class}).

\item[ templated statistical:] vector realizer, vector RV, statistical IP (and options), MH solver (and options), statistical FP (and options), MC solver (and options), sequence statistical options;
%\item Vector RV (concatenation)
%\item Statistical IP (and options)
%\item Metropolis Hastings solver (and options)
%\item Statistical FP (and options)
%\item MC solver (and options)
%\item Sequence statistical options
\item[ miscellaneous:] C and FORTRAN interfaces.
\end{description}



\subsection{Environment Class (and Options)}

%
The Environment class sets up the environment underlying the use of the QUESO library by an executable.
The constructor of the environment class requires a communicator, the name of an options input file,
and the eventual prefix of the environment in order for the proper options to be read (multiple environments can coexist, as explained further below).
The environment class has four primary tasks:
\begin{enumerate}
\item assign rank numbers, other than the world rank, to nodes participating in a parallel job,
\item provide MPI communicators for generating a sequence of vectors in a distributed way,
\item provide functionality to read options from the options input file (whose name is passed in the constructor of this environment class), and
\item open output files for messages that would otherwise be written to the screen (one output file per allowed rank is opened and allowed ranks can be specified through the options input file).
 
\end{enumerate}



Let $S \geqslant 1$ be the number of problems a QUESO environment will be handling at the same time, in parallel.
$S$ has default value of $1$ and is an option read by QUESO from the input file provided by the user.
The QUESO environment class manages five types of communicators, referred to as:
\begin{description}
\item[{\it world} :] MPI\_WORLD\_COMM;
\item[{\it full} :] communicator passed to the environment constructor, of size $F$ and usually equal to the world communicator;
\item[{\it sub} :] communicator of size $F/S$ that contains the number of MPI nodes necessary to solve a statistical IP or a statistical FP;
\item[{\it self} :] MPI\_SELF\_COMM, of size 1; and
\item[{\it inter0} :] communicator of size $S$ formed by all MPI nodes that have subrank 0 in their respective subcommunicators.
\end{description}


A {\it subenvironment} in QUESO is the smallest collection of processors necessary for the proper run of the model code.
An {\it environment} in QUESO is the collection of all subenvironments, if there is more than one subenvironment.
So, for instance, if the model code requires 16 processors to run and the user decides to run 64 Markov chains in parallel,
then the environment will consist of a total of $F=1024$ processors and $S=64$ subenvironments, each subenvironment with $F/S=16$ processors.
Any given computing node in a QUESO run has potentially five different ranks.
Each subenvironment is assigned a subid varying from $0$ (zero) to $S-1$, and is able to handle a statistical IP and/or a statistical FP.
That is, each subenvironment is able to handle a {\it sub} Markov chain (a sequence) of vectors and/or a {\it sub} MC sequence of output vectors.
The {\it sub} sequences form an unified sequence in a distributed way.
QUESO takes care of the unification of results for the application programming and for output files.


Figure \ref{fig-env-class} depicts class diagram for the environment class; and Figure  \ref{fig-env-options-class} displays environment options class.%, pages \pageref{fig-env-class} and \pageref{fig-env-options-class}),

\begin{figure}[!hp]
\centering
\includegraphics[scale=0.40,clip=true]{figs/uqEnvironment}
\vspace*{-8pt}
\caption{The class diagram for the environment class.}
\label{fig-env-class}
\end{figure}

\begin{figure}[hp]
\centering
\includegraphics[scale=0.40,clip=true]{figs/uqEnvironmentOptions}
\vspace*{-8pt}
\caption{The environment options class.}
\label{fig-env-options-class}
\end{figure}

\subsubsection{Input File Options}
The input file options for a QUESO environment is presented in Table \ref{tab-env-options}.
\begin{table}[htpb]
\begin{center}
\caption{Input file options for a QUESO environment.}\label{tab-env-options}

\begin{tabular}{l c  m{6cm}}
\toprule
Option name                      &  Default  value & Description \\
\midrule\midrule

\ttfamily $\langle$PREFIX$\rangle$env\_help                &         &             \\
\midrule
\ttfamily$\langle$PREFIX$\rangle$env\_numSubEnvironments   &   1     &  Number of subenvironments                \\ %UQ_ENV_NUM_SUB_ENVIRONMENTS_ODV
\midrule
\ttfamily$\langle$PREFIX$\rangle$env\_subDisplayFileName   &  ``.''  & Output filename for subscreen writing     \\ %UQ_ENV_SUB_DISPLAY_FILE_NAME_ODV
\midrule
\ttfamily$\langle$PREFIX$\rangle$env\_subDisplayAllowAll   &   0     & Allow all subEnvs to write to output file  \\ %UQ_ENV_SUB_DISPLAY_ALLOW_ALL_ODV
\midrule
\ttfamily$\langle$PREFIX$\rangle$env\_subDisplayAllowedSet & ``''    & Subenvironments that will write to output file      \\ %UQ_ENV_SUB_DISPLAY_ALLOWED_SET_ODV
\midrule
\ttfamily$\langle$PREFIX$\rangle$env\_displayVerbosity     &   0     & Set verbosity				           \\ %UQ_ENV_DISPLAY_VERBOSITY_ODV
\midrule
\ttfamily$\langle$PREFIX$\rangle$env\_syncVerbosity        &   0     & Set syncronized verbosity            \\ %UQ_ENV_SYNC_VERBOSITY_ODV
\midrule
\ttfamily$\langle$PREFIX$\rangle$env\_seed                 &   0     & Set seed           \\ %UQ_ENV_SEED_ODV
%
% TODO add the following options:
% (m_option_platformName.c_str(),         po::value<std::string >()->default_value(UQ_ENV_PLATFORM_NAME_ODV),           "platform name")""
% (m_option_identifyingString.c_str(),    po::value<std::string >()->default_value(UQ_ENV_IDENTIFYING_STRING_ODV),      "identifying string")""
% (m_option_checkingLevel.c_str(),        po::value<unsigned int>()->default_value(UQ_ENV_CHECKING_LEVEL_ODV),          "set checking level")  0          
\bottomrule
\end{tabular}
\end{center}
\end{table}

  

%\clearpage
\subsection{Vector}


The Vector class handles handles all the vector operations carried out in QUESO, and its class diagram is presented in Figure \ref{fig-vector-class}.
Vector class has two derived classes: \verb+uqGslVectorClass+ and \verb+uqTrilinosVectorClass+. \verb+uqGslVectorClass+ is based on the GSL vector structure; whereas \verb+uqTrilinosVectorClass+ is based on Trilinos Epetra vector structure.


\begin{figure}[!hpt]
\centering
\includegraphics[scale=0.40,clip=true]{figs/uqVector}
\vspace*{-8pt}
\caption{ The class diagram for the vector class.}
\label{fig-vector-class}
\end{figure}



\subsection{Matrix}


The Matrix class handles handles all the matrix operations carried out in QUESO, and its class diagram is presented in Figure \ref{fig-matrix-class}. Analogously to the Vector class case,
Matrix class has two derived classes: \verb+uqGslMatrixClass+ and \verb+uqTrilinosMatrixClass+. \verb+uqGslMatrixClass+ is based on the GSL matrix structure; whereas \verb+uqTrilinosMatrixClass+ is based on Trilinos Epetra matrix structure.


\begin{figure}[!hp]
\centering
\includegraphics[scale=0.40,clip=true]{figs/uqMatrix}
\vspace*{-8pt}
\caption{The class diagram for the matrix class.}
\label{fig-matrix-class}
\end{figure}


%\clearpage
\section{Templated Basic Classes}
The classes in this group are: Vector sets, subsets and spaces (Section \ref{sec:vector-set-space}), scalar and vector function classes (Section \ref{sec:scalar-vector-function}), and scalar and vector sequences (Section \ref{sec:scalar-vector-sequence}).

These classes constitute the core entities necessary for the formal
mathematical definition and description of other entities, such as
random variables, Bayesian solutions of inverse problems, sampling algorithms and chains.

%\clearpage


\subsection{Vector Set  and Vector Space Classes}\label{sec:vector-set-space}
%
The vector set class is fundamental for the proper handling of many mathematical entities.
Indeed, the definition of a scalar function such as $\pi:\mathbf{B}\subset\mathbb{R}^n\rightarrow\mathbb{R}$ requires the
specification of the domain $\mathbf{B}$, which is a {\it subset} of the {\it vector space} $\mathbb{R}^n$, which is itself a {\it set}. Indeed, 
 SIPs need a likelihood routine $\pi_{\text{like}}:\mathbb{R}^n\rightarrow\mathbb{R}_+$,
and SFPs need a QoI routine $\mathbf{q}:\mathbb{R}^n\rightarrow\mathbb{R}^m$; the \textit{sets} $\mathbb{R}^n$, $\mathbb{R}^m$, etc., are {\it vector spaces}.

The relationship among the classes set, subset and vector space is sketched in Figure \ref{fig-vector-space-subset-classes}.
%
An attribute of the {\it subset} class is the {\it vector space} which it belongs to, and in fact a reference to a vector space is required by the constructor of the subset class. An example of this case is the definition of a scalar function such as $\pi:\mathbf{B}\subset\mathbb{R}^n\rightarrow\mathbb{R}$, which requires the specification of the domain $\mathbf{B}$, which is a {\it subset} of the {\it vector space} $\mathbb{R}^n$, which is itself a {\it set}.
The power of an object-oriented design is clearly featured here.
The {\it intersection} subset derived class is useful for handling a posterior PDF \eqref{eq-Bayes-solution},
since its domain is the intersection of the domain of the prior PDF with the domain of the likelihood function.

\begin{figure}[hp]
\centering
\includegraphics[scale=0.40,clip=true]{figs/uqVectorSet}
\vspace*{-8pt}
\caption{The class diagram for vector set, vector subset and vector space classes.}
\label{fig-vector-space-subset-classes}
\end{figure}

%\clearpage
\subsection{Scalar Function and Vector Function Classes}\label{sec:scalar-vector-function}

Joint PDF, marginal PDF, and CDF are all examples of scalar functions present in statistical problems. 
QUESO currently supports basic PDFs such as uniform and Gaussian.
See Diagram~\ref{fig-scalar-function-class} for the scalar function class.

The handling of vector functions within QUESO is as easy. Indeed,
The definition of a vector function $\mathbf{q}:\mathbf{B}\subset\mathbb{R}^n\rightarrow\mathbb{R}^m$ requires only the extra specification of the image vector space $\mathbb{R}^m$. Diagram \ref{fig-vector-function-class} represents the vector function class.
\begin{figure}[htpb]
\centering
\includegraphics[scale=0.40,clip=true]{figs/uqScalarFunction}
\vspace{-8pt}
\caption{The class diagram for the scalar function class.}
\label{fig-scalar-function-class}
\end{figure}

\begin{figure}[htpb]
\centering
\includegraphics[scale=0.40,clip=true]{figs/uqVectorFunction}
\vspace{-8pt}
\caption{The class diagram for the vector function class.}
\label{fig-vector-function-class}
\end{figure}

%\clearpage

\subsection{Scalar Sequence and Vector Sequence Classes}\label{sec:scalar-vector-sequence}
%
The scalar sequence class contemplates {\it scalar} samples generated by an algorithm, as well as operations that can
be done over them, e.g., calculation of means, variances, and convergence indices.
%% such as Geweke and Brooks-Gelman.
Similarly, the vector sequence class contemplates {\it vector} samples and operations such as means, correlation matrices and covariance matrices.

Figures \ref{fig-scalar-sequence-class} and \ref{fig-vector-sequence-class} display the class diagram for the scalar sequence  and vector sequence classes, respectively.

\begin{figure}[htpb]
\centering
\includegraphics[scale=0.40,clip=true]{figs/uqScalarSequence}
\vspace{-8pt}
\caption{The class diagram for the scalar sequence class.}
\label{fig-scalar-sequence-class}
\end{figure}

\begin{figure}[htpb]
\centering
\includegraphics[scale=0.40,clip=true]{figs/uqVectorSequence}
\vspace{-8pt}
\caption{The class diagram for the vector sequence class.}
\label{fig-vector-sequence-class}
\end{figure}



%\clearpage
\section{Templated Statistical Classes}

The classes in this group are: Vector realizer, Vector random variable, Statistical inverse problem (and options), Metropolis-Hastings solver (and options), Statistical forward problem (and options), Monte Carlo solver (and options), and Sequence statistical options.

For QUESO, a statistical inverse problem has two input entities, a prior random variable and
a likelihood routine, and one output entity, the posterior random variable, as shown in Figure \ref{fig-sip-queso}.
%
Similarly, a statistical forward problem for QUESO has two input entities, a input random variable and
a qoi routine, and one output entity, the output random variable, as shown in Figure \ref{fig-sfp-queso}.


\subsection{Vector Realizer Class}
%
A {\it realizer} is an object that, simply put, contains a {\it realization()} operation that returns a sample of a vector RV.
QUESO currently supports basic realizers such as uniform and Gaussian.
It also contains a {\it sequence realizer} class for storing samples of a MH algorithm, for instance.




\subsection{Vector Random Variable Class}
%
Vector RVs are expected to have two basic functionalities:
compute the value of its PDF at a point, and generate realizations following such PDF.
The joint PDF and vector realizer classes allow a straightforward definition and manipulation of vector RVs.
QUESO currently supports basic vector RVs such as uniform and Gaussian.
A derived class called {\it generic vector RV} allows QUESO to store the solution of an statistical IP:
a {\it Bayesian joint PDF} becomes the PDF of the posterior RV, while a {\it sequence vector realizer} becomes the realizer of the same posterior RV.
QUESO also allows users to form new RVs through the concatenation of existing RVs.


\begin{figure}[htpb]
\centering
\includegraphics[scale=0.40,clip=true]{figs/uqVectorRandomVariable}
\vspace{-8pt}
\caption{The class diagram for the vector random variable class.}
\label{fig-vector-rv-class}
\end{figure}



%\clearpage
\subsection{Statistical Inverse Problem (and Options)}
Similarly to its mathematical concepts, a SIP in QUESO also expects two input entities, a prior RV and a likelihood routine, and one output entity, the posterior RV.
The SIP is represented in QUESO through the templated class \verb+uqStatisticalInverseProblemClass<P_V,P_M>+.
One important characteristic of the QUESO design is that it  separates `what the problem is' from `how the problem is solved'.
The prior and the posterior RV are instances of the \verb+uqBaseVectorRvClass<V,M>+ class, while
the likelihood function is an instance of the \verb+uqBaseScalarFunctionClass<V,M>+ class.

The solution of a SIP is computed by calling the \verb+solveWithBayesMetropolisHastings(...)+ member function of the \verb+uqStatisticalInverseProblemClass<P_V,P_M>+ class.
Upon return from a solution operation, the posterior RV is available through the \verb+postRv()+ member function.

%For the specific problem for inferring the acceleration of gravity (Section \ref{sec:sip}),
%the uniform prior is chosen as an instance of \verb+uqUniformeVectorRvClass<V,M>+,
%the likelihood is an instance of \verb+uqGenericScalarFunctionClass<V,M>+, and
%the posterior is an instance of  \verb+uqGenericVectorRvClass<V,M>+. 

\begin{figure}[htp!]
\centering
\includegraphics[scale=0.40,clip=true]{figs/uqSip}
\vspace{-8pt}
\caption{The statistical inverse problem class. It implements the representation in Figure~\ref{fig-sip-queso}.}
\label{fig-sip-class}
\end{figure}

\begin{figure}[htp!]
\centering
\includegraphics[scale=0.40,clip=true]{figs/uqSipOptions}
\vspace{-8pt}
\caption{The statistical inverse problem options class.}
\label{fig-sip-options-class}
\end{figure}

\begin{table}[htpb]
\begin{center}
\caption{Input file options for a QUESO statistical inverse problem.}
\ttfamily

\begin{tabular}{l|c|c}
\toprule
\rmfamily Option name                    & \rmfamily Default  Value & \rmfamily Description \\
\midrule\midrule
$\langle$PREFIX$\rangle$ip\_help                 &         &             \\

$\langle$PREFIX$\rangle$ip\_computeSolution      &         &             \\

$\langle$PREFIX$\rangle$ip\_dataOutputFileName   &         &             \\

$\langle$PREFIX$\rangle$ip\_dataOutputAllowedSet &         &             \\
\bottomrule
\end{tabular}
\end{center}

\label{tab-sip-options}
\end{table}



%\clearpage
\subsection{Metropolis-Hastings Solver (and Options)}

\begin{figure}[htpb]
\centering
\includegraphics[scale=0.40,clip=true]{figs/uqMetropolisHastingsSG}
\vspace*{-8pt}
\caption{The Metropolis-Hastings sequence generator class.}
\label{fig-metropolis-hastings-solver-class}
\end{figure}

\begin{figure}[htpb]
\centering
\includegraphics[scale=0.40,clip=true]{figs/uqMetropolisHastingsSGOptions}
\vspace*{-8pt}
\caption{The Metropolis-Hastings sequence generator options class.}
\label{fig-metropolis-hastings-options-class}
\end{figure}

\begin{table}[htpb]
\begin{center}
\caption{Input file options for a QUESO Metropolis-Hastings solver.}
\ttfamily
\begin{tabular}{l|c|c}
\toprule
\rmfamily Option Name                                    & \rmfamily Default Value& \rmfamily Description \\
\midrule\midrule
$\langle$PREFIX$\rangle$mh\_help                                &         &             \\

$\langle$PREFIX$\rangle$mh\_dataOutputFileName                  &         &             \\

$\langle$PREFIX$\rangle$mh\_dataOutputAllowedSet                &         &             \\

$\langle$PREFIX$\rangle$mh\_rawChain\_dataInputFileName         &         &             \\

$\langle$PREFIX$\rangle$mh\_rawChain\_size                      &         &             \\

$\langle$PREFIX$\rangle$mh\_rawChain\_generateExtra             &         &             \\

$\langle$PREFIX$\rangle$mh\_rawChain\_displayPeriod             &         &             \\

$\langle$PREFIX$\rangle$mh\_rawChain\_measureRunTimes           &         &             \\

$\langle$PREFIX$\rangle$mh\_rawChain\_dataOutputFileName        &         &             \\

$\langle$PREFIX$\rangle$mh\_rawChain\_dataOutputAllowedSet      &         &             \\

$\langle$PREFIX$\rangle$mh\_rawChain\_computeStats              &         &             \\

$\langle$PREFIX$\rangle$mh\_filteredChain\_generate             &         &             \\

$\langle$PREFIX$\rangle$mh\_filteredChain\_discardedPortion     &         &             \\

$\langle$PREFIX$\rangle$mh\_filteredChain\_lag                  &         &             \\

$\langle$PREFIX$\rangle$mh\_filteredChain\_dataOutputFileName   &         &             \\

$\langle$PREFIX$\rangle$mh\_filteredChain\_dataOutputAllowedSet &         &             \\

$\langle$PREFIX$\rangle$mh\_filteredChain\_computeStats         &         &             \\

$\langle$PREFIX$\rangle$mh\_displayCandidates                   &         &             \\

$\langle$PREFIX$\rangle$mh\_putOutOfBoundsInChain               &         &             \\

$\langle$PREFIX$\rangle$mh\_tk\_useLocalHessian                 &         &             \\

$\langle$PREFIX$\rangle$mh\_tk\_useNewtonComponent              &         &             \\

$\langle$PREFIX$\rangle$mh\_dr\_maxNumExtraStages               &         &             \\

$\langle$PREFIX$\rangle$mh\_dr\_scalesForExtraStages            &         &             \\

$\langle$PREFIX$\rangle$mh\_am\_initialNonAdaptInterval         &         &             \\

$\langle$PREFIX$\rangle$mh\_am\_adaptInterval                   &         &             \\

$\langle$PREFIX$\rangle$mh\_am\_eta                             &         &             \\

$\langle$PREFIX$\rangle$mh\_am\_epsilon                         &         &             \\
\bottomrule
\end{tabular}
\end{center}
\label{tab-metropolis-hastings-options}
\end{table}

%\clearpage
\subsection{Statistical Forward Problem (and Options)}

A SFP in QUESO also has two input entities, the input (parameter) RV and a QoI function, and one output entity, the QoI RV. 
The SIP is represented through the templated class \verb+uqStatisticalForwardProblemClass<P_V,P_M>+.
The input RV and the output QoI RV are instances of the \verb+uqBaseVectorRvClass<V,M>+ class, while
the QoI function is an instance of \verb+uqBaseVectorFunctionClass<P_V,P_M,Q_V,Q_M>+.
In the template parameters, the prefix \verb+P_+ refers to the parameters, whereas the prefix \verb+Q_+ refers to the QoIs.

In order to find the solution of a SFP, one must call the \verb+solveWithMonteCarlo(...)+ member function of the \verb+uqStatisticalForwardProblemClass<P_V,P_M>+ class.
Upon return from a solution operation, the QoI RV is available through the \verb+qoiRv()+ member function.
%it provides both CDFs of QoI components through the operation \verb+qoiRv().unifiedCdf()+,  which returns an instance of the class \verb+uqBaseVectorCdfClass<Q_V,Q_M>+, and
%a vector realizer through the operation \verb+qoiRv().realizer()+, which returns an instance of the class \verb+uqBaseVectorRealizerClass<Q_V,Q_M>+.

\begin{figure}[htpb]
\centering
\includegraphics[scale=0.40,clip=true]{figs/uqSfp}
\vspace*{-8pt}
\caption{The statistical forward problem class. It implements the representation in Figure \ref{fig-sfp-queso}.}
\label{fig-sfp-class}
\end{figure}

\begin{figure}[htpb]
\centering
\includegraphics[scale=0.40,clip=true]{figs/uqSfpOptions}
\vspace*{-8pt}
\caption{The statistical forward problem options class.}
\label{fig-sfp-options-class}
\end{figure}

\begin{table}[htpb]
\caption{Input file options for a QUESO statistical forward problem.}
\vspace{-8pt}
\ttfamily
\begin{center}
\begin{tabular}{l|c|c}
\toprule
 \rmfamily Option Name                     & \rmfamily Default Value& \rmfamily Description \\
\midrule
$\langle$PREFIX$\rangle$fp\_help                 &         &             \\
%\hline
$\langle$PREFIX$\rangle$fp\_computeSolution      &         &             \\
%\hline
$\langle$PREFIX$\rangle$fp\_computeCovariances   &         &             \\
%\hline
$\langle$PREFIX$\rangle$fp\_computeCorrelations  &         &             \\
%\hline
$\langle$PREFIX$\rangle$fp\_dataOutputFileName   &         &             \\
%\hline
$\langle$PREFIX$\rangle$fp\_dataOutputAllowedSet &         &             \\
\bottomrule
\end{tabular}
\end{center}

\label{tab-sfp-options}
\end{table}

%\clearpage
\subsection{Monte Carlo Solver (and Options)}

\begin{figure}[htpb]
\centering
\includegraphics[scale=0.40,clip=true]{figs/uqMonteCarloSG}
\vspace*{-8pt}
\caption{{\color{red}{The Monte Carlo sequence generator class}}.}
\label{fig-monte-carlo-solver-class}
\end{figure}

\begin{figure}[htpb]
\centering
\includegraphics[scale=0.40,clip=true]{figs/uqMonteCarloSGOptions}
\vspace*{-8pt}
\caption{{\color{red}{The Monte Carlo sequence generator options class}}.}
\label{fig-monte-carlo-options-class}
\end{figure}

\begin{table}[htpb]
\begin{center}
\caption{Input file options for a QUESO statistical forward problem.}
\ttfamily
\begin{tabular}{|l|c|c|}
\toprule
\rmfamily Option Name     & \rmfamily Default Value &  \rmfamily Description \\
\midrule\midrule
$\langle$PREFIX$\rangle$mc\_help                        &         &             \\
%\hline
$\langle$PREFIX$\rangle$mc\_dataOutputFileName          &         &             \\
%\hline
$\langle$PREFIX$\rangle$mc\_dataOutputAllowedSet        &         &             \\
%\hline
$\langle$PREFIX$\rangle$mc\_pseq\_dataOutputFileName     &         &             \\
%\hline
$\langle$PREFIX$\rangle$mc\_pseq\_dataOutputAllowedSet   &         &             \\
%\hline
$\langle$PREFIX$\rangle$mc\_pseq\_computeStats           &         &             \\
% \hline
$\langle$PREFIX$\rangle$mc\_qseq\_dataInputFileName      &         &             \\
% \hline
$\langle$PREFIX$\rangle$mc\_qseq\_size                   &         &             \\
% \hline
$\langle$PREFIX$\rangle$mc\_qseq\_displayPeriod          &         &             \\
% \hline
$\langle$PREFIX$\rangle$mc\_qseq\_measureRunTimes        &         &             \\
% \hline
$\langle$PREFIX$\rangle$mc\_qseq\_dataOutputFileName     &         &             \\
% \hline
$\langle$PREFIX$\rangle$mc\_qseq\_dataOutputAllowedSet   &         &             \\
% \hline
$\langle$PREFIX$\rangle$mc\_qseq\_computeStats           &         &             \\
\bottomrule
\end{tabular}
\end{center}
\label{tab-monte-carlo-options}
\end{table}

%\clearpage
\subsection{Options for Statistical Analysis of Sequences}

\begin{figure}[htpb]
\centering
\includegraphics[scale=0.40,clip=true]{figs/uqSequenceStatisticalOptions}
\vspace*{-8pt}
\caption{{\color{red}{The sequence statistical options class}}.}
\label{fig-seq-statistical-options-class}
\end{figure}

\begin{table}[htpb]
\begin{center}
\caption{Input file options for a the statistical analysis of sequences.}
\ttfamily
\begin{tabular}{l|c|c}
\toprule
\rmfamily Option Name  & \rmfamily Default Value & \rmfamily Description \\
\midrule\midrule
$\langle$PREFIX$\rangle$stats\_help                      &         &             \\
% \hline
$\langle$PREFIX$\rangle$stats\_initialDiscardedPortions  &         &             \\
% \hline
\hline
$\langle$PREFIX$\rangle$stats\_bmm\_run                   &         &             \\
% \hline
$\langle$PREFIX$\rangle$stats\_bmm\_lengths               &         &             \\
% \hline
$\langle$PREFIX$\rangle$stats\_bmm\_display               &         &             \\
% \hline
$\langle$PREFIX$\rangle$stats\_bmm\_write                 &         &             \\
% \hline
\hline
$\langle$PREFIX$\rangle$stats\_fft\_compute               &         &             \\
% \hline
$\langle$PREFIX$\rangle$stats\_fft\_paramId               &         &             \\
% \hline
$\langle$PREFIX$\rangle$stats\_fft\_size                  &         &             \\
% \hline
$\langle$PREFIX$\rangle$stats\_fft\_testInversion         &         &             \\
% \hline
$\langle$PREFIX$\rangle$stats\_fft\_write                 &         &             \\
% \hline
\hline
$\langle$PREFIX$\rangle$stats\_psd\_compute               &         &             \\
% \hline
$\langle$PREFIX$\rangle$stats\_psd\_numBlocks             &         &             \\
% \hline
$\langle$PREFIX$\rangle$stats\_psd\_hopSizeRatio          &         &             \\
% \hline
$\langle$PREFIX$\rangle$stats\_psd\_paramId               &         &             \\
% \hline
$\langle$PREFIX$\rangle$stats\_psd\_write                 &         &             \\
% \hline
\hline
$\langle$PREFIX$\rangle$stats\_psdAtZero\_compute         &         &             \\
% \hline
$\langle$PREFIX$\rangle$stats\_psdAtZero\_numBlocks       &         &             \\
% \hline
$\langle$PREFIX$\rangle$stats\_psdAtZero\_hopSizeRatio    &         &             \\
% \hline
$\langle$PREFIX$\rangle$stats\_psdAtZero\_display         &         &             \\
% \hline
$\langle$PREFIX$\rangle$stats\_psdAtZero\_write           &         &             \\
% \hline
\hline
$\langle$PREFIX$\rangle$stats\_geweke\_compute            &         &             \\
% \hline
$\langle$PREFIX$\rangle$stats\_geweke\_naRatio            &         &             \\
% \hline
$\langle$PREFIX$\rangle$stats\_geweke\_nbRatio            &         &             \\
% \hline
$\langle$PREFIX$\rangle$stats\_geweke\_display            &         &             \\
% \hline
$\langle$PREFIX$\rangle$stats\_geweke\_write              &         &             \\
% \hline
\hline
$\langle$PREFIX$\rangle$stats\_autoCorr\_computeViaDef    &         &             \\
% \hline
$\langle$PREFIX$\rangle$stats\_autoCorr\_computeViaFft    &         &             \\
% \hline
$\langle$PREFIX$\rangle$stats\_autoCorr\_secondLag        &         &             \\
% \hline
$\langle$PREFIX$\rangle$stats\_autoCorr\_lagSpacing       &         &             \\
% \hline
$\langle$PREFIX$\rangle$stats\_autoCorr\_numLags          &         &             \\
% \hline
$\langle$PREFIX$\rangle$stats\_autoCorr\_display          &         &             \\
% \hline
$\langle$PREFIX$\rangle$stats\_autoCorr\_write            &         &             \\
% \hline
\hline
$\langle$PREFIX$\rangle$stats\_meanStacc\_compute         &         &             \\
% \hline
\hline
$\langle$PREFIX$\rangle$stats\_hist\_compute              &         &             \\
% \hline
$\langle$PREFIX$\rangle$stats\_hist\_numInternalBins      &         &             \\
% \hline
\hline
$\langle$PREFIX$\rangle$stats\_cdfStacc\_compute          &         &             \\
% \hline
$\langle$PREFIX$\rangle$stats\_cdfStacc\_numEvalPositions &         &             \\
% \hline
\hline
$\langle$PREFIX$\rangle$stats\_kde\_compute               &         &             \\
% \hline
$\langle$PREFIX$\rangle$stats\_kde\_numEvalPositions      &         &             \\
% \hline
\hline
$\langle$PREFIX$\rangle$stats\_covMatrix\_compute         &         &             \\
% \hline
$\langle$PREFIX$\rangle$stats\_corrMatrix\_compute        &         &             \\
\bottomrule
\end{tabular}
\end{center}
\label{tab-seq-statistical-options}
\end{table}


%\clearpage
\section{Miscellaneous Classes and Routines}

%\clearpage
\section{Interface Classes}



\subsection{Using Other C++ Classes in the Library}

https://svn.ices.utexas.edu/repos/pecos/turbulence/IncompRansCal/calDataLevModUncert/README


\subsection{Input and Output Files}


