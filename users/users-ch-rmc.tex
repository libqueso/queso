\chapter{The Realization of a Markov Chain}\label{ch-rmc}
\thispagestyle{headings}
\markboth{Chapter \ref{ch-rmc}: The Realization of a Markov Chain}{Chapter \ref{ch-rmc}: The Realization of a Markov Chain}

In this chapter we detail the
Metropolis-Hastings (MH),
Delayed Rejection (DR),
Adaptive Metropolis (AM) and
DRAM algorithms.

While we indicated r.v.'s and their realizations by $\mathbf{Y}_k$ and $\mathbf{y}_k$ in Chapter \ref{ch-mcmc},
in this chapter we will primarily use supper scripts and the greek letter $\theta$ to indicate r.v.'s and their realizations, e.g. $\boldsymbol{\Theta}^{(k)}$ and $\boldsymbol{\theta}^{(k)}$.
Sub scripts will refer to components $\theta_i^{(k)}$ of $\boldsymbol{\theta}^{(k)}\in\mathbb{R}^m$.
In this way, we return to the initial terminology used in Chapter \ref{ch-int}.

\section{Preliminary Definitions}

In what follows we also assume that
we are given
\begin{equation}\label{eq-pi-target}
\text{a target probability density }\pi_{\text{target}}:\mathbb{R}^n\rightarrow [0,1],
\end{equation}
\begin{equation}\label{eq-y0}
\text{a relization }\boldsymbol{\theta}^{(0)}\text{ of a r.v. }\boldsymbol{\Theta}^{(0)}:\Omega\rightarrow\mathbb{R}^n,
\end{equation}
\begin{equation}\label{eq-q1}
\text{a proposal kernel (see Subsection \ref{subsc-proposal-kernel}) }q_1:\mathbb{R}^n\times \mathbb{R}^n\rightarrow [0,1]\text{ and}
\end{equation}
\begin{equation}\label{eq-n-chain}
\text{an integer }n_{\text{chain}}\geqslant 1,
\end{equation}
and that
we want to construct a {\it realization} of
a homogeneous Markov chain of size $n_{\text{chain}}$ with a transition probability kernel $K:\mathbb{R}^n\rightarrow [0,1]$
such that the limit density of $K$ is $\pi_{\text{target}}$.
An example of a target is the density equal (up to a multiplicative constant) to the posterior density discussed
at the end of Section \ref{sc-intro-qoi}.

\subsection{Detailed Balance Equation and Reversibility}
$~$\\

\subsection{The Proposal Transition Probability Kernel $q$}\label{subsc-proposal-kernel}
$~$\\

\section{Algorithms}\label{sc-rmc-algs}

Before we proceed to the actual algorithms, a definition is necessary.
Given a probability density $\pi:\mathbb{R}^n\rightarrow [0,1]$,
an integer $k\geqslant 1$ and
$k$ proposal transition probability kernels
\begin{equation*}
q_i:\underbrace{\mathbb{R}^n\times\ldots\times\mathbb{R}^n}_{(i+1)\text{ times}}\rightarrow [0,1],\quad 1\leqslant i\leqslant k,
\end{equation*}
we recursively define
\begin{equation}\label{eq-alphas}
\alpha_i:\underbrace{\mathbb{R}^n\times\ldots\times\mathbb{R}^n}_{(i+1)\text{ times}}\rightarrow [0,1],\quad 1\leqslant i\leqslant k,
\end{equation}
by setting
\begin{equation*}
\alpha_1(\boldsymbol{\theta},\mathbf{c}^{(1)}) = \text{ min}
\left\{
1,\frac
{\pi(\mathbf{c}^{(1)})q_1(\mathbf{c}^{(1)},\boldsymbol{\theta})}
{\pi(\boldsymbol{\theta})q_1(\boldsymbol{\theta},\mathbf{c}^{(1)})}
\right\},
\end{equation*}
and, for $i>1$,
\begin{equation*}
\alpha_i(\boldsymbol{\theta},\mathbf{c}^{(1)},\ldots,\mathbf{c}^{(i)}) = \text{ min}
\left\{
1,\frac
{\pi(\mathbf{c}^{(i)})}
{\pi(\boldsymbol{\theta})}
\cdot q_{\text{fraction}}
\cdot \alpha_{\text{fraction}}
\right\}.
\end{equation*}
where
the expressions $q_{\text{fraction}}$ and $\alpha_{\text{fraction}}$ are given by
\begin{equation*}
q_{\text{fraction}}=
\frac
{q_1(\mathbf{c}^{(i)},\mathbf{c}^{(i-1)})}
{q_1(\boldsymbol{\theta},\mathbf{c}^{(1)})}
\frac
{q_2(\mathbf{c}^{(i)},\mathbf{c}^{(i-1)},\mathbf{c}^{(i-2)})}
{q_2(\boldsymbol{\theta},\mathbf{c}^{(1)},\mathbf{c}^{(2)})}
\ldots
\frac
{q_i(\mathbf{c}^{(i)},\mathbf{c}^{(i-1)},\ldots,\mathbf{c}^{(1)},\boldsymbol{\theta})}
{q_i(\boldsymbol{\theta},\mathbf{c}^{(1)},\ldots,\mathbf{c}^{(i-1)},\mathbf{c}^{(i)})}
\end{equation*}
and
\begin{equation*}
\alpha_{\text{fraction}}=
\frac
{[1-\alpha_1(\mathbf{c}^{(i)},\mathbf{c}^{(i-1)})]}
{[1-\alpha_1(\boldsymbol{\theta},\mathbf{c}^{(1)})]}
\frac
{[1-\alpha_2(\mathbf{c}^{(i)},\mathbf{c}^{(i-1)},\mathbf{c}^{(i-2)})]}
{[1-\alpha_2(\boldsymbol{\theta},\mathbf{c}^{(1)},\mathbf{c}^{(2)})]}
\ldots
\frac
{[1-\alpha_{i-1}(\mathbf{c}^{(i)},\mathbf{c}^{(i-1)},\ldots,\mathbf{c}^{(1)})]}
{[1-\alpha_{i-1}(\boldsymbol{\theta},\mathbf{c}^{(1)},\ldots,\mathbf{c}^{(i-1)})]}.
\end{equation*}
It should be emphasized that $\boldsymbol{\theta}$ does {\it not} appear on the numerator of $\alpha_{\text{fraction}}$.

\subsection{The Metropolis-Hastings (MH) Algorithm}%\label{subsc-rmc-mh-alg}

The Metropolis-Hastings algorithm proceeds as follows:
\begin{enumerate}
\item for ($k=0$; $k < (n_{\text{chain}}-1)$; ++$k$) \{
\item $\quad$/* Perform $(k+1)$-th iteration in order to obtain $\boldsymbol{\theta}^{(k+1)}$ */
\item $\quad$generate a candidate $\mathbf{c}$ from $q_1(\boldsymbol{\theta}^{(k)},\cdot)$;
\item $\quad$accept = false;
\item $\quad$outOfBounds = ($\theta_i < \theta_{i,\text{min}}$ or $\theta_{i,\text{max}} < \theta_i$ for some $i$);
\item $\quad$if (outOfBounds == false) \{
\item $\quad\quad$compute the acceptance ratio $\alpha=\alpha_1(\boldsymbol{\theta}^{(k)},\mathbf{c})$ (see \eqref{eq-alphas});
\item $\quad\quad$draw $t$ from the uniform distribution on $[0,1]$;
\item $\quad\quad$if ($t\leqslant \alpha$) \{ $\boldsymbol{\theta}^{(k+1)}=\mathbf{c}$; accept = true; \}
\item $\quad$\}
\item $\quad$if (accept == false) $\boldsymbol{\theta}^{(k+1)}=\boldsymbol{\theta}^{(k)}$;
\item \} /* end for */.
\end{enumerate}

\subsection{The Delayed Rejeciton (DR) Algorithm}%\label{subsc-rmc-dr-alg}

Given
\begin{equation}\label{eq-ne}
n_e\geqslant 1
\end{equation}
extra proposal kernels $q_i:S\times S\rightarrow [0,1]$, $i=2,\ldots,n_e+1$,
the Delayed Rejection algorithm proceeds as follows:
\begin{enumerate}
\item for ($k=0$; $k < (n_{\text{chain}}-1)$; ++$k$) \{
\item $\quad$/* Perform $(k+1)$-th iteration in order to obtain $\boldsymbol{\theta}^{(k+1)}$ */
\item $\quad$generate a candidate $\mathbf{c}^{(1)}$ from $q_1(\boldsymbol{\theta}^{(k)},\cdot)$;
\item $\quad$accept = false; $s=1$; /* ``s'' stands for ``stage id'' */
\item $\quad$outOfBounds = ($\theta_i < \theta_{i,\text{min}}$ or $\theta_{i,\text{max}} < \theta_i$ for some $i$);
\item $\quad$if (outOfBounds == false) \{
\item $\quad\quad$compute the acceptance ratio $\alpha=\alpha_1(\boldsymbol{\theta}^{(k)},\mathbf{c}^{(1)})$ (see \eqref{eq-alphas});
\item $\quad\quad$draw $t$ from the uniform distribution on $[0,1]$;
\item $\quad\quad$if ($t\leqslant \alpha$) \{ $\boldsymbol{\theta}^{(k+1)}=\mathbf{c}^{(s)}$; accept = true; \}
\item $\quad\quad$else while ((accept == false) \&\& ($i\leqslant n_e$)) \{
\item $\quad\quad\quad$/* Extra stages trying extra candidates $\mathbf{c}^{(2)},\mathbf{c}^{(3)},\ldots,\mathbf{c}^{(n_e+1)}$ (maximum) */
\item $\quad\quad\quad$generate a candidate $\mathbf{c}^{(s+1)}$ from $q_i(\boldsymbol{\theta}^{(k)},\mathbf{c}^{(1)},\ldots,\mathbf{c}^{(s)})$;
\item $\quad\quad\quad$compute the acceptance ratio $\alpha=\alpha_i(\boldsymbol{\theta}^{(k)},\mathbf{c}^{(1)},\ldots,\mathbf{c}^{(s)})$ (see \eqref{eq-alphas});
\item $\quad\quad\quad$draw $t$ from the uniform distribution on $[0,1]$;
\item $\quad\quad\quad$if ($t\leqslant \alpha$) \{ $\boldsymbol{\theta}^{(k+1)}=\mathbf{c}^{(s)}$; accept = true; \}
\item $\quad\quad\quad$$i\leftarrow i+1$;
\item $\quad\quad$\} /* end while */
\item $\quad$\}
\item $\quad$if (accept == false) $\boldsymbol{\theta}^{(k+1)}=\boldsymbol{\theta}^{(k)}$;
\item \} /* end for */.
\end{enumerate}
So, if a rejection happens, the algorithm yet tries to find a suitable canditate under the same $(k+1)$-th iteration.
If $n_e=0$ then the DR algorithm becomes the MH algorithm of the previous subsection.

\subsection{The Adaptive Metropolis (AM) Algorithm}\label{subsc-rmc-am-alg}
Given
\begin{equation}\label{eq-C0}
\text{a }n\times n\text{ covariance matrix }C_0,
\end{equation}
\begin{equation}\label{eq-eta}
\text{a real value }\eta >0,
\end{equation}
\begin{equation}\label{eq-epsilon}
\text{a real value }\epsilon>0,
\end{equation}
\begin{equation}\label{eq-t0}
\text{an integer }t_0>0,
\end{equation}
\begin{equation}\label{eq-p0}
\text{an integer }p_0>0,
\end{equation}
and, for $k > 0$ and any vectors $\boldsymbol{\theta}^{(0)},\boldsymbol{\theta}^{(1)},\ldots,\boldsymbol{\theta}^{(k)}$, the expression
\begin{equation}\label{eq-emperical-cov}
\text{Cov}(\boldsymbol{\theta}^{(0)},\boldsymbol{\theta}^{(1)},\ldots,\boldsymbol{\theta}^{(k)})\equiv,
\end{equation}
the Adaptive Metropolis algorithm proceeds as follows:
\begin{enumerate}
\item for ($k=0$; $k < (n_{\text{chain}}-1)$; ++$k$) \{
\item $\quad$/* Perform $(k+1)$-th iteration in order to obtain $\boldsymbol{\theta}^{(k+1)}$ */
\item $\quad$generate a candidate $\mathbf{c}$ from $q_1(\boldsymbol{\theta}^{(k)},\cdot)$;
\item $\quad$compute the acceptance ratio $\alpha=\alpha_1(\boldsymbol{\theta}^{(k)},\mathbf{c})$ (see \eqref{eq-alphas});
\item $\quad$draw $t$ from the uniform distribution on $[0,1]$;
\item $\quad$if ($t\leqslant \alpha$) $\boldsymbol{\theta}^{(k+1)}=\mathbf{c}$;
\item $\quad$else $\boldsymbol{\theta}^{(k+1)}=\boldsymbol{\theta}^{(k)}$;
\item \} /* end for */.
\end{enumerate}

\begin{sidewaystable}[h]
\begin{tabular}{|c||c|c||c|c||c|c|}
\hline
 Position to         & \multicolumn{2}{c||}{Metropolis-Hastings (``MH'')}                & \multicolumn{2}{c||}{Delayed Rejection (``DR'')}                         & \multicolumn{2}{c|}{Adaptive Metropolis (``AM'')}          \\
\cline{2-7}
 be computed         & candidate               & resulting position                      & candidate              & resulting position                              & candidate        & resulting position                      \\
(iteration)          &                         &                                         & (stage)                &                                                 &                  &                                         \\
\hline
\hline
$\boldsymbol{\theta}^{(1)}$       & $\mathbf{c}$ w/ $q$              & $\boldsymbol{\theta}^{(0)}$ or $\mathbf{c}$ w/ $\alpha_1$       & $\mathbf{c}^{(1)}$ w/ $q_1$     & $\boldsymbol{\theta}^{(0)}$ or $\mathbf{c}^{(1)}$ w/ $\alpha_1$       & $\mathbf{c}$ w/ $g_0$     & $\boldsymbol{\theta}^{(0)}$ or $\mathbf{c}$ w/ $\alpha_1$       \\
\cline{4-5}
                     &                         &                                         & $\mathbf{c}^{(2)}$ w/ $q_2$     & $\boldsymbol{\theta}^{(0)}$ or $\mathbf{c}^{(2)}$ w/ $\alpha_2$       &                  &                                         \\
\cline{4-5}
                     &                         &                                         & $\vdots$               & $\vdots$                                        &                  &                                         \\
\cline{4-5}
                     &                         &                                         & \multicolumn{2}{l||}{until candidate accepted}                           &                  &                                         \\
                     &                         &                                         & \multicolumn{2}{l||}{or maximum stages achieved}                         &                  &                                         \\
\hline 
\hline
$\boldsymbol{\theta}^{(2)}$       & $\mathbf{c}$ w/ $q$              & $\boldsymbol{\theta}^{(1)}$ or $\mathbf{c}$ w/ $\alpha_1$       & $\mathbf{c}^{(1)}$ w/ $q_1$     & $\boldsymbol{\theta}^{(1)}$ or $\mathbf{c}^{(1)}$ w/ $\alpha_1$       & $\mathbf{c}$ w/ $g_0$     & $\boldsymbol{\theta}^{(1)}$ or $\mathbf{c}$ w/ $\alpha_1$       \\
\cline{4-5}
                     &                         &                                         & $\mathbf{c}^{(2)}$ w/ $q_2$     & $\boldsymbol{\theta}^{(1)}$ or $\mathbf{c}^{(2)}$ w/ $\alpha_2$       &                  &                                         \\
\cline{4-5}
                     &                         &                                         & $\vdots$               & $\vdots$                                        &                  &                                         \\
\cline{4-5}
                     &                         &                                         & \multicolumn{2}{l||}{until candidate accepted}                           &                  &                                         \\
                     &                         &                                         & \multicolumn{2}{l||}{or maximum stages achieved}                         &                  &                                         \\
\hline
\hline
$\vdots$             & $\vdots$                & $\vdots$                                & $\vdots$               & $\vdots$                                        & $\vdots$         & $\vdots$                                \\
\hline
\hline
$\boldsymbol{\theta}^{(t_0)}$     & $\mathbf{c}$ w/ $q$              & $\boldsymbol{\theta}^{(t_0-1)}$ or $\mathbf{c}$ w/ $\alpha_1$     & $\mathbf{c}^{(1)}$ w/ $q_1$     & $\boldsymbol{\theta}^{(t_0-1)}$ or $\mathbf{c}^{(1)}$ w/ $\alpha_1$     & $\mathbf{c}$ w/ $g_0$     & $\boldsymbol{\theta}^{(t_0-1)}$ or $\mathbf{c}$ w/ $\alpha_1$     \\
\cline{4-5}
                     &                         &                                         & $\vdots$               & $\vdots$                                        &                  &                                         \\
\hline
\hline
$\boldsymbol{\theta}^{(t_0+1)}$   & $\mathbf{c}$ w/ $q$              & $\boldsymbol{\theta}^{(t_0)}$ or $\mathbf{c}$ w/ $\alpha_1$   & $\mathbf{c}^{(2)}$ w/ $q_1$     & $\boldsymbol{\theta}^{(t_0)}$ or $\mathbf{c}^{(1)}$ w/ $\alpha_1$   & $\mathbf{c}$ w/ $g_{t_0+1}$     & $\boldsymbol{\theta}^{(t_0)}$ or $\mathbf{c}$ w/ $\alpha_1$   \\
\cline{4-5}
                     &                         &                                         & $\vdots$               & $\vdots$                                        &                  &                                         \\
\hline
\hline
$\vdots$             & $\vdots$                & $\vdots$                                & $\vdots$               & $\vdots$                                        & $\vdots$         & $\vdots$                                \\
\hline
\hline
$\boldsymbol{\theta}^{(t_0+p_0+1)}$ & $\mathbf{c}$ w/ $q$              & $\boldsymbol{\theta}^{(t_0+p_0)}$ or $\mathbf{c}$ w/ $\alpha_1$ & $\mathbf{c}^{(2)}$ w/ $q_1$     & $\boldsymbol{\theta}^{(t_0+p_0)}$ or $\mathbf{c}^{(1)}$ w/ $\alpha_1$ & $\mathbf{c}$ w/ $g_{t_0+p_0+1}$     & $\boldsymbol{\theta}^{(t_0+p_0)}$ or $\mathbf{c}$ w/ $\alpha_1$ \\
\cline{4-5}
                     &                         &                                         & $\vdots$               & $\vdots$                                        &                  &                                         \\
\hline
\hline
$\vdots$             & $\vdots$                & $\vdots$                                & $\vdots$               & $\vdots$                                        & $\vdots$         & $\vdots$                                \\
\hline
\hline
$\boldsymbol{\theta}^{(t_0+2p_0+1)}$& $\mathbf{c}$ w/ $q$              & $\boldsymbol{\theta}^{(t_0+2p_0)}$ or $\mathbf{c}$ w/ $\alpha_1$& $\mathbf{c}^{(2)}$ w/ $q_1$     & $\boldsymbol{\theta}^{(t_0+2p_0)}$ or $\mathbf{c}^{(1)}$ w/ $\alpha_1$& $\mathbf{c}$ w/ $g_{t_0+2p_0+1}$     & $\boldsymbol{\theta}^{(t_0+2p_0)}$ or $\mathbf{c}$ w/ $\alpha_1$\\
\cline{4-5}
                     &                         &                                         & $\vdots$               & $\vdots$                                        &                  &                                         \\
\hline
\hline
$\vdots$             & $\vdots$                & $\vdots$                                & $\vdots$               & $\vdots$                                        & $\vdots$         & $\vdots$                                \\
\hline
\end{tabular}
\caption{Overview of three algorithms for the generation of a {\it realization} of a Markov chain 
$\{\boldsymbol{\theta}^{(0)},\boldsymbol{\theta}^{(1)},\ldots\}$
: Metropolis-Hastings, Delayed Rejection and Adaptive Metropolis.
Detailed explanations are given in Section \ref{sc-rmc-algs}.
}
\label{tab-dram}
\end{sidewaystable}

\subsection{The DRAM Algorithm}%\label{subsc-rmc-dram-alg}
$~$\\

\subsection{The Gibbs Sampler}

To be explained in future versions of the documentation.

\section{Chain Statistics}\label{sc-rmc-chain-stats}

After the realization
\begin{equation}\label{eq-markov-chain-2}
\{\boldsymbol{\theta}^{(0)},\boldsymbol{\theta}^{(1)},\ldots\,\boldsymbol{\theta}^{(n_{\text{chain}}-1)}\}
\end{equation}
of a chain,
many different statistical evaluations can be performed over the parameter sequence
\begin{equation}\label{eq-markov-chains-for-each-specific-parameter}
\{{\theta}_i^{(0)},{\theta}_i^{(1)},\ldots,{\theta}_i^{(n_{\text{chain}}-1)}\},
\end{equation}
for each parameter $1\leqslant i\leqslant n_{\text{sip}}$.

\subsection{Rejections}

\subsection{Measurements of Central Tendency}

\subsection{Measurements of Dispersion}

\subsection{Convergence Diagnostics}

Transient phase/bias.

Equilibrium phase/autocorrelation \cite{BrRo98} \cite{So96}.

\subsection{Kernel Density Estimators}

After computing a chain with the user requested $n_{\text{chain}}$ samples, the toolkit might displays the following values, depending on the user options expressed in the input file:
\begin{equation}\label{eq-dram-chain-stats-rej}
\text{rej}(\theta_i) = 0,
\end{equation}
%
\begin{equation}\label{eq-dram-chain-stats-oor}
\text{oor}(\theta_i) = 0,
\end{equation}
%
\begin{equation}\label{eq-dram-chain-stats-mean}
\langle\theta_i\rangle = 0,
\end{equation}
%
\begin{equation}\label{eq-dram-chain-stats-std}
\sigma(\theta_i) = 0,
\end{equation}
%
\begin{equation}\label{eq-dram-chain-stats-bm}
\sigma_{\text{bm}}(\theta_i) = 0,
\end{equation}
%
\begin{equation}\label{eq-dram-chain-stats-gew}
\text{gew}(\theta_i) = 0,
\end{equation}
\begin{equation}\label{eq-dram-chain-stats-tau-int}
\tau_{\text{int}}(\theta_i) = 0,
\end{equation}
and
\begin{equation}\label{eq-dram-chain-stats-iqr}
\text{iqr}(\theta_i) = 0,
\end{equation}
where


